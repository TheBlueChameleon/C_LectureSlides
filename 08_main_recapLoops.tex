% =========================================================================== %
% Yes. This is a document.

\documentclass[
	ngerman,
	aspectratio=169,
	table
]{beamer}

% =========================================================================== %
% Theme
\usepackage{scrlfile}
	\ReplacePackage{beamerthemeSHUR}{./sty/beamerthemeSHUR}
	\ReplacePackage{beamerinnerthemefancy}{./sty/beamerinnerthemefancy}
	\ReplacePackage{beamerouterthemedecolines}{./sty/beamerouterthemedecolines}
	\ReplacePackage{beamercolorthemechameleon}{./sty/beamercolorthemechameleon}

\usetheme[
	pageofpages=/,
	bullet=circle,
	titleline=true,
	alternativetitlepage=true,
	watermark="",
	watermarkheight=0px,
	watermarkheightmult=0
	]
{SHUR}

% =================================================================================================== %
% the usual stuff

\usepackage[utf8]{inputenc}
\usepackage[T1]{fontenc}
\usepackage{babel}
\usepackage{lmodern}
\usepackage{microtype}
\usepackage{csquotes}

\usepackage{tabularx}
\usepackage{booktabs}
\usepackage{multirow}

\usepackage{color, colortbl}
\usepackage{xcolor}

\usepackage{tabto}

\usepackage{minted}
	\usemintedstyle{friendly}

\usepackage{tikz}
	\usetikzlibrary{positioning}
	\usetikzlibrary{matrix}
	\usetikzlibrary{shapes.geometric}
	\usetikzlibrary{backgrounds}
	\usetikzlibrary{calc}
	\usetikzlibrary{decorations.pathreplacing}
	\tikzstyle{every picture}+=[remember picture] 
\usepackage{adjustbox}

\usepackage[most]{tcolorbox}
	\tcbsetforeverylayer
		{colback=cyan!10!white,
		 colframe=cyan!75!black,
		 arc=0pt,
		 outer arc=0pt
		}
	\newtcolorbox{codebox}[1][Code]
		{colback=black!5!white,
		 colframe=blue!40!black,
		 title=#1,
		 leftupper=6mm
		}
	\newtcolorbox{cmdbox}[1][Kommandozeilen-Befehl]
		{colback=black,
		 coltext=white,
		 fontupper=\ttfamily ,
		 colframe=blue!40!black,
		 title=#1,
		 outer arc=0pt
		}
	\newtcolorbox{warnbox}[1][Beachte]
		{colback=black!5!white,
		 colframe=red!40!black,
		 title=#1
		}
	\newtcolorbox{hintbox}[1][Tipp]
		{colback=black!5!white,
		 colframe=green!40!black,
		 title=#1
		}%
	\newenvironment{itembox}[1][]
		{\begin{tcolorbox}[title=#1]\begin{itemize}}%
		{\end{itemize}\end{tcolorbox}}
	\newtcolorbox{doublebox}[1][.3]
		{righthand width=#1\linewidth,
		 sidebyside,
		 sidebyside gap=6mm,
		 sidebyside align=center,
		 lower separated=false}

%==============================================================================%
% GLOBAL MACROS

\newcommand*{\zB}{z.\,B. }
\newcommand*{\ua}{u.\,a. }
\newcommand*{\ie}{d.\,h. }			%% dh is already a defined macro. Couldn't find out what it does.
\newcommand*{\idR}{i.\,d.\,R. }

\newcommand*{\tabcrlf}{\\ \midrule}			% actually still allows for optional argument

\newcommand*{\inC}[1]{\mintinline{c}{#1}}

\newcommand*{\thus}{\ensuremath{\rightarrow}\xspace}
\newcommand*{\Thus}{\ensuremath{\Rightarrow}\xspace}

% =========================================================================== %

\author{Stefan Hartinger}
\title{Programmieren in C und C++}
\subtitle{Kursteil 8: Wiederholung: Schleifen und Funktionen}
\institute{Universität Regensburg, Fakultät Physik}
\date{Blockkurs Herbst 2021}

% =========================================================================== %

\begin{document}
% =========================================================================== %
% --------------------------------------------------------------------------- %

\begin{frame}[t,plain]
\titlepage
\end{frame}

% =========================================================================== %

\begin{frame}[fragile]{Schleifen}
%
\begin{itemize}
\item \mintinline{c}{while}: Wiederhole, solange eine Bedingung erfüllt ist
\item Bedingung: Beliebiger Ausdruck, der sich zu einem Wahrheitswert auswerten lässt.
\item Wahrheitswert: Entweder 0 oder nicht 0
\item Daher auch: \mintinline{c}{while (variable)}
\item Achtung: \emph{Alles} ist Zahl
\item Achtung bei Operatoren: Doppeltes Gleichheitszeichen \texttt{==} ! 
\end{itemize}
%
\end{frame}

% =========================================================================== %

\begin{frame}[fragile]
%
\begin{tcbraster}[raster columns=2, raster equal height, nobeforeafter]
%
\begin{codebox}[Beispiel: \texttt{strlen}, equal height group=A]
\begin{minted}[fontsize=\scriptsize, linenos]{c}
#include <stdio.h>

int strlen(char * string) {
  int Stelle = 0;
  
  if (string) {
    while(string[Stelle]) {Stelle++;}
    return Stelle;
    
  } else {
    return -1;
  }
}
\end{minted}
\end{codebox}
%
\begin{codebox}[...Fortsetzung, equal height group=A]
\begin{minted}[fontsize=\scriptsize, linenos]{c}
int main () {
  printf("Länge des leeren Strings: "
         "%d\n", strlen("")
  );
  
  char string[] = "exurb1a!";
  printf("%s: "
         "%d\n", string, strlen(string)
  );
  
  string[3] = 0;
  printf("%s: "
         "%d\n", string, strlen(string)
  );
}
\end{minted}
\end{codebox}
\end{tcbraster}
%
%
\begin{cmdbox}[Ausgabe]
\begin{minted}[fontsize=\scriptsize]{text}
Länge des leeren Strings: 0      exurb1a!: 8     exu: 3
\end{minted}
\end{cmdbox}
%
\end{frame}

% =========================================================================== %

\begin{frame}{Zählschleifen}
%
\begin{itemize}
\item \mintinline{c}{for}-Schleifen für \enquote{abzählbare} Mengen
\item \enquote{Für jedes \texttt{x} mache...}
\item 80\% der Fälle: Durchlaufe Array-Indizes
\item Daher auch meist: \mintinline{c}{int}-Zählvariable, beginnend bei 0
\item Gefühlt feste Form, auch wenn andere Konstruktionen möglich sind
\item Zweiter Block: Wahrheitswert. Daher kann auch hier Vergleichsoperator manchmal entfallen
\end{itemize}
%
\end{frame}

% =========================================================================== %

\begin{frame}[fragile]
%
\tcbset{width=.495\linewidth, on line}
%
\begin{codebox}[Beispiel: \texttt{strlen} mit \texttt{for}, equal height group=B]
\begin{minted}[fontsize=\scriptsize, linenos]{c}
int strlen(char * string) {
  if (string) {
    int Stelle = 0;
    
    for (; string[Stelle]; Stelle ++) {}
    return Stelle;
    
  } else {
    return -1;
  }
}
\end{minted}
\end{codebox}
%
\begin{hintbox}[Tipp, equal height group=B]
\small
Solche Konstruktionen können effiziente und elegante Lösungen darstellen; in den meisten Fällen \enquote{stolpert} man aber beim Lesen über die \mintinline{c}{for}-Zeile. In der Regel ist die \enquote{Standard Form}:
\begin{center}
\mintinline{c}{for(int i=0; i<N; i++)}
\end{center}
zu bevorzugen.
\end{hintbox}
%
%
\end{frame}

% =========================================================================== %

\begin{frame}[fragile]{\mintinline{c}{break} und \mintinline{c}{continue}}
%
\begin{columns}[T]
\column{.4\linewidth}
\begin{itemize}
\item Beide Statements: implizites \mintinline{c}{goto}
\item \mintinline{c}{break}: Hinter die Schleifen-Klammer (\})
\item \mintinline{c}{continue}: Vor die Schleifen-Klammer (\{)
\item Funktioniert nur über eine Ebene hinweg
\item Wenn mehrere Schleifen gleichzeitig übersprungen werden sollen: \mintinline{c}{goto} oder Flags
	(Variablen mit Wahrheitswert)
\end{itemize}
%
\column{.57\linewidth}
\begin{codebox}[Beispiel: Verschachtelte Schleifen verlassen]
\begin{minted}[fontsize=\scriptsize, linenos]{c}
#include <stdio.h>

int main () {
  for   (int i=1; i<100; i++) {
    for (int j=1; j<100; j++) {
      if (i + j > 150) {
        goto label_end;
      }
      printf("%d + %d = %d", i, j, i+j);
    }
  }
  
  label_end:
}
\end{minted}
\end{codebox}
%
\end{columns}
%
\end{frame}

% =========================================================================== %

\begin{frame}[fragile]
%
\begin{columns}[T]
\column{.45\linewidth}
\begin{Large}
{\mintinline{c}{for}: Variation des Iterationsschritts}
\vspace{6pt}
\end{Large}
%
\begin{itemize}
\item Häufige Abweichung von der \enquote{Standard-Form}:
\item Andere Inkremente als \texttt{+1}
\item Trivial: Nur geradzahlige Elemente  (\texttt{i += 2})
\item Schritte Variabler Breite \\
	(\texttt{i += var})
\item \texttt{var} kann auch innerhalb der Schleife geändert werden!
\end{itemize}
%
\column{.45\linewidth}
\begin{codebox}[Beispiel: Variable Schrittweite, equal height group=B]
\begin{minted}[fontsize=\scriptsize, linenos]{c}
#include <stdio.h>

int main () {
  for (int i=1; i<100; i += i) {
    printf("%d\n", i);
  }
}
\end{minted}
\end{codebox}
%
\begin{cmdbox}[Ausgabe, equal height group=B]
\begin{minted}[fontsize=\scriptsize]{text}
1
2
4
8
16
32
64
\end{minted}
\end{cmdbox}
%
\end{columns}
%
\end{frame}

% =========================================================================== %

\begin{frame}[fragile]
%
\begin{columns}[T]
\column{.45\linewidth}
\begin{Large}
{Schleifen und Scopes}
\vspace{6pt}
\end{Large}
%
\begin{itemize}
\item \emph{Alle} Strukturen, die \{geschweifte Klammern\} benutzen, bilden eigene Scopes
\item Also auch \mintinline{c}{for}, \mintinline{c}{while}, \mintinline{c}{do..while},
	  \mintinline{c}{if}, \mintinline{c}{switch}, \ldots
\item \mintinline{c}{for}: Sogar zwei Ebenen: Schleifenkopf und -körper
\item Variablen da deklarieren, wo gebraucht
\item Namen eindeutig vergeben
\end{itemize}
%
\column{.45\linewidth}
\begin{codebox}[Beispiel: Variable Schrittweite]
\begin{minted}[fontsize=\scriptsize, linenos]{c}
#include <stdio.h>

int main () {
  int i = 1;
  
  for (int i = 0; i < 3; i++) {
    int i = 2;
    printf("i=%d\n", i);
  }
  printf("i=%d\n", i);
}
\end{minted}
\end{codebox}
%
\begin{cmdbox}[Ausgabe]
\begin{minted}[fontsize=\scriptsize]{text}
i=2
i=2
i=2
i=1
\end{minted}
\end{cmdbox}
%
\end{columns}
%
\end{frame}

% =========================================================================== %

\begin{frame}[fragile]
%
(Lösung zur Kartenspiel-Aufgabe)
%
\end{frame}

% =========================================================================== %
\end{document}