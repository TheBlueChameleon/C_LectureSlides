% =========================================================================== %

\begin{frame}[t,plain]
\titlepage
\end{frame}

% =========================================================================== %

\begin{frame}[fragile]{\texttt{valgrind} -- Test auf unzulässige Speicherzugriffe}
%
\begin{itemize}
\item Kommandozeilentool für Linux. Parameter: Zu analysierendes Programm
\item Erlaubt weitere Parameter: Weitergabe an zu analysierendes Programm
\item Programm wird normal gestartet, aber von valgrind \enquote{überwacht}.
\item Programm läuft dadurch 20-30 mal langsamer
\item Ausgabe von unzulässigen Lese- und Schreibzugriffen auf der Konsole
\item Analyse über Speichernutzung (fehlende \texttt{free}s)
\item \url{http://valgrind.org/docs/manual/index.html}
\end{itemize}
%
\end{frame}

% =========================================================================== %

\begin{frame}[fragile]
%
\begin{cmdbox}[Beispiel: Kommandozeile -- Fehlerfreier Code]
\begin{minted}[fontsize=\scriptsize]{text}
gcc -std=c11 -Wall -Wextra -Wpedantic mycode.c -lm
valgrind ./mycode command line parameters


==22888== 
==22888== HEAP SUMMARY:
==22888==     in use at exit: 0 bytes in 0 blocks
==22888==   total heap usage: 0 allocs, 0 frees, 0 bytes allocated
==22888== 
==22888== All heap blocks were freed -- no leaks are possible
==22888== 
==22888== For counts of detected and suppressed errors, rerun with: -v
==22888== ERROR SUMMARY: 0 errors from 0 contexts (suppressed: 0 from 0)
\end{minted}
\end{cmdbox}
%
\end{frame}

% =========================================================================== %

\begin{frame}[fragile]
%
\tcbset{width=.495\linewidth, on line, height=7.8cm}
%
\begin{codebox}[Beispiel: Code mit Fehlern]
\begin{minted}[fontsize=\scriptsize, linenos]{c}
#include <stdlib.h>

void f() {
  int * x = malloc(10 * sizeof(int));
  x[10] = 0;
}                    

int main(void) {
  f();
  return 0;
}

// problem 1: heap block overrun
// problem 2: memory leak -- x not freed
\end{minted}
\tiny \url{http://valgrind.org/docs/manual/quick-start.html}
\end{codebox}
%
\begin{cmdbox}[valgrind-Ausgabe]
\begin{minted}[fontsize=\scriptsize]{text}
...
Command: ./A02_valgrind

Invalid write of size 4
  at 0x400544: f (in A02_valgrind)
  by 0x400555: main (A02_valgrind)
...
HEAP SUMMARY:
    in use at exit: 40 bytes in 1 blocks
  total heap usage: 1 allocs, 0 frees, 
                    40 bytes allocated

LEAK SUMMARY:
  definitely lost: 40 bytes in 1 blocks
  indirectly lost: 0 bytes in 0 blocks
    possibly lost: 0 bytes in 0 blocks
  still reachable: 0 bytes in 0 blocks
       suppressed: 0 bytes in 0 blocks
Rerun with --leak-check=full to see 
    details of leaked memory
\end{minted}
\end{cmdbox}
%
\end{frame}

% =========================================================================== %

\begin{frame}{\texttt{gdb} -- Gnu Debugger}
%
\begin{columns}[T]
\column{.5\linewidth}
\begin{itemize}
\item Linux-Tool, auch für Windows und Mac
\item Variablen im Laufenden Programm abprüfen
\item Kompilieren mit \texttt{gcc}-Debug-Option \texttt{-g}
	\begin{itemize}
	\item \texttt{gcc -std=c11 -Wall -g myProg.c}
	\end{itemize}
\item Eigene Kommandozeilen-Umgebung \texttt{gdb}
	\begin{itemize}
	\item \texttt{file <executable>} -- Datei festlegen
	\item \texttt{run <command line parameters>} -- Programm starten
	\item \texttt{print <variable>} -- Wert ausgeben
	\item \texttt{break <function>} -- Haltepunkt festlegen
	\end{itemize}
\end{itemize}
%
\column{.5\linewidth}
\begin{itemize}
\item \texttt{gdb}-Kommando \texttt{run}: Programm läuft normal bis zum Absturz oder bis zum Erreichen
	von definierten Haltepunkten
\item \texttt{gdb}-Kommandos erlauben dann Lesen des Zustands
\item \url{https://web.eecs.umich.edu/~sugih/pointers/summary.html}
\end{itemize}
\end{columns}
%
\end{frame}

% =========================================================================== %

\begin{frame}[fragile]
%
\begin{cmdbox}[Anwendungsbeispiel: gdb]
\begin{minted}[fontsize=\scriptsize]{text}
gcc -std=c11 E0X_recursion.c -g3 -lm -o recursion
gdb
...
(gdb) file recursion
Reading symbols from recursion...fertig.
break 190
Haltepunkt 1 at 0x555555554f0c: file E0X_recursion.c, line 190.
(gdb) run
Starting program: /home/blue-chameleon/Documents/Uni/30 SHK/04 - C-Kurs 2019-09/
  04 - Codes/recursion
Breakpoint 2, showtree (startDir=0x0) at E0X_recursion.c:191
191       if (!startDir) {
(gdb) print indent_level
$1 = 0
(gdb) print flag_free_startDir
$2 = 1
continue
...
[Inferior 1 (process 8953) exited normally]
\end{minted}
\end{cmdbox}
%
\end{frame}

% =========================================================================== %

\begin{frame}{C++ -- Eine mächtige Sprache}
%
\begin{warnbox}
C++ ist eine eigenständige Sprache! Trotz großer Ähnlichkeiten sollten die beiden nie \enquote{unter einen Hut gesteckt} werden.
\end{warnbox}
%
\begin{itemize}
\item Formell: Weiterentwicklung von C -- vgl. Inkrement-Operator
\item Eigener Compiler \texttt{g++} ersetzt \texttt{gcc}.
\item \texttt{gcc} und \texttt{g++} für beide Sprachen geeignet, jedoch unterschiedlich optimiert
\item Viele neue Konzepte, von denen hier nur einige exemplarisch gezeigt werden
\item \emph{Gigantische} Sammlung an Libraries
\item Support: \url{http://en.cppreference.com} bzw. \url{http://de.cppreference.com}
\item UR bietet eigenen Kurs an:
	\href{http://www.physik.uni-regensburg.de/studium/it/c++kurs/}
	{$\rightarrow$ Programmieren mit C++ und der Qt-Bibliothek $\leftarrow$}
\end{itemize}
%
\end{frame}

% =========================================================================== %

\begin{frame}[fragile]
%
\begin{columns}[T]
\column{.5\linewidth}
\begin{Large}
{C vs. C++ -- Unterschiede}
\vspace{10pt}
\end{Large}
%
\begin{itemize}
\item Stream-Konzepte
	\begin{itemize}
	\item Bereitstellung eines Puffers
	\item User schreibt in Puffer
	\item System gibt gepufferete Daten zu geeigneter Zeit an Geräte wie Bildschirm weiter
	\item \texttt{printf} $\rightarrow$ \texttt{cout}
	\item \texttt{scanf}  $\rightarrow$ \texttt{cin}
	\end{itemize}
\item Function Overloading
\item Namespaces
\item Klassen -- Objektorientierung
\item Templates
\end{itemize}
%
\column{.5\linewidth}
\begin{codebox}[Beispiel: Hello World in C++]
\begin{minted}[fontsize=\scriptsize, linenos]{c++}
#include <iostream>

using namespace std;

int main (void) {
  int a, b, c;
  
  cout << "Hello world" << endl;
  cout << "Please enter 3 ints:" << endl;
  cin  >> a >> b >> c;
  cout << "You entered:" << endl;
  cout << a << b << b << endl;
  return 0;
}
\end{minted}
\end{codebox}
\end{columns}
%
\end{frame}

% =========================================================================== %

\begin{frame}[fragile]
%
\begin{columns}[T]
\column{.5\linewidth}
\begin{Large}
{Default Arguments und Function Overloading}
\vspace{10pt}
\end{Large}
%
\begin{itemize}
\item Default Arguments
	\begin{itemize}
	\item Parameter beim Aufruf \enquote{auslassbar}
	\item Wenn ausgelassen: Zuweisung von Default-Wert
	\end{itemize}
\item Function Overloading
	\begin{itemize}
	\item Mehrere Funktionen mit gleichem Namen
	\item Zweck: Selbe Aufgabe für unterschiedliche Parametertypen oder Rückgabetypen erfüllen
	\item Code der einzelnen Funktionen unabhängig
	\end{itemize}

\end{itemize}
\column{.5\linewidth}
\begin{codebox}[Beispiel: Default Arguments]
\begin{minted}[fontsize=\scriptsize, linenos]{c++}
int getBiggestIdx(
   int * arr, int N, 
   int first=0, int last=-1);
...
cout << getBiggestIdx(arr1, N, 5);
cout << getBiggestIdx(arr2, N,  , 7);
\end{minted}
\end{codebox}
%
\begin{codebox}[Beispiel: Function Overloading]
\begin{minted}[fontsize=\scriptsize, linenos]{c++}
double distance(double x1, double y1,
                double x2, double y2);

double distance(double x1, double y1, 
                double z1,
                double x2, double y2, 
                double z2);
\end{minted}
\end{codebox}
\end{columns}
%
\end{frame}

% =========================================================================== %

\begin{frame}[fragile]
%
\begin{columns}[T]
\column{.5\linewidth}
\begin{Large}
{Strings in C++}
\vspace{10pt}
\end{Large}
%
\begin{itemize}
\item Eigener \enquote{Datentyp} (eigentlich: \emph{Klasse})
\item Baut auf C-String auf
\item Kommt mit einer großen Zahl von \emph{Methoden}: Routinen, die auf Strings angewandt werden können
\item Library \texttt{string}
\item \texttt{at} -- Elemente lesen oder schreiben mit \emph{boundary check}
\item \texttt{size}, \texttt{length} -- Zeichen bis zum ersten \texttt{NULL}-Char
\item \texttt{capacity} -- Größe des zugrunde liegenden \mintinline{c}{char}-Arrays
\end{itemize}
%
\column{.5\linewidth}
\begin{codebox}[Beispiel: Hello World in C++]
\begin{minted}[fontsize=\scriptsize, linenos]{c++}
#include <iostream>
#include <string>
using namespace std;

int main (void) {
  string myStr = "some funky words";
  
  // output entire string
  cout << myStr << endl;
  
  // output 4th char -- boundary check
  cout << myStr.at(3);  // or myStr[3]
  
  // change 9th char
  myStr.at(8) = 'n'
  return 0;
}
\end{minted}
\end{codebox}
\end{columns}
%
\end{frame}

% =========================================================================== %

\begin{frame}[fragile]
%
\begin{columns}[T]
\column{.45\linewidth}
\begin{Large}
{Strings in C++ (Fortsetzung)}
\vspace{10pt}
\end{Large}
%
\begin{itemize}
\item \texttt{clear} -- String leeren
\item \texttt{insert(where, count, what)} -- Anderen String einfügen
\item \texttt{find\_first\_of(what)} -- Zeichen in einem String finden
\item und viele mehr
\item[$\rightarrow$] \url{http://en.cppreference.com/w/cpp/string/basic_string}
\end{itemize}
%
\column{.55\linewidth}
\begin{codebox}[Beispiel: Insert Strings]
\begin{minted}[fontsize=\scriptsize, linenos]{c++}
#include <iostream>
using namespace std;

int main (void) {
  string s = "something Hieronymus";
  
  s.insert(
    s.find_first_of(' ') + 1,
    "funny about "
  );
  
  cout << s;
  
  return 0;
}
\end{minted}
\end{codebox}
\end{columns}
%
\end{frame}

% =========================================================================== %

\begin{frame}{Warum immer noch C lernen?}
%
\tcbset{width=.495\linewidth, on line}
%
\begin{tcolorbox}[title=Zitat, height=4cm]
%
\vspace{5pt}
\begin{quotation}
In C++ it’s harder to shoot yourself in the foot, but when you do, you blow off your whole leg.
\end{quotation}
\vspace{-11pt}
%
\begin{flushright}
\footnotesize \href{https://www.youtube.com/watch?v=9QKHg8wj4MA}{Bjarne Stroustrup, Entwickler von C++}
\end{flushright}
%
\end{tcolorbox}
%
\includegraphics[width=.495\linewidth]{./gfx/cppFirst}
%
\begin{itemize}
\item Bild: Fehlermeldung bei meinem ersten Kompilierdurchgang zu \emph{Insert Strings}.
\item Fehlermeldungen erstrecken sich über mehr als eine Bildschirmseite
\item Fehler war unnötige Referenz auf Startpunkt
\item[$\Rightarrow$] Erst wer C sicher beherrscht, wird C++ nutzen können
\end{itemize}
%
\end{frame}

% =========================================================================== %

\begin{frame}{Klassen -- Objektorientierung}
%
\begin{columns}[T]
\column{.5\linewidth}
\begin{itemize}
\item Weiterentwicklung von \mintinline{c}{structs}
\item Funktionen zu Klasse zugeordnet
\item Zugriff auf Records der Klasse über Schlüsselwort \mintinline{c++}{this}: Pointer auf Klassenobjekt
\item Constructors und Destructors
	\begin{itemize}
	\item Automatischer Funktionsaufruf beim Erstellen und oder Freigeben
	\item Betrifft auch Verlassen des Scopes
	\end{itemize}
% CTOR, DTOR
\end{itemize}
%
\column{.5\linewidth}
\begin{itemize}
\item Zugriffsbeschränkung: \mintinline{c++}{private}, \mintinline{c++}{public}
	\begin{itemize}
	\item \mintinline{c++}{public} -- Zugriff auf Record von überall -- wie aus C bekannt
	\item \mintinline{c++}{private} -- Zugriff nur von Klassenmethoden aus
	\item Sinn: Verhindere \enquote{unsinnige} Zustände (Bsp: Matrix-Klasse: Record Zeilen passt nicht 
		zum  Dateninhalt)
	\end{itemize}
\end{itemize}
\end{columns}
%
\end{frame}

% =========================================================================== %

\begin{frame}[fragile]
%
\tcbset{width=.495\linewidth, on line, height=7.8cm}
%
\begin{codebox}[Beispiel: Klasse in C++ ...]
\begin{minted}[fontsize=\scriptsize, linenos]{c++}
#include <iostream>
using namespace std;

class Mat {
public:
   int getRows();
   int getCols();

   double getElement(int row, int col);

   // constructors
   Mat ();
   Mat (int rows, int cols);

   // destructor
   ~Mat ();
private:
   int rows, cols;
   double * data;
};
\end{minted}
\end{codebox}
%	
\begin{codebox}[... Fortsetzung ...]
\begin{minted}[fontsize=\scriptsize, linenos, firstnumber=last]{c++}
Mat::Mat (int rows, int cols) {
   this->rows = rows;
   this->cols = cols;
   this->data = new double[rows * cols];
}

Mat::~Mat () {delete this->data;}

int Mat::getRows() {return this->rows;}

double Mat::getElement(int r, int c) {
   if (
      (r >= this->rows) || (r < 0) ||
      (c >= this->cols) || (c < 0)
   ) {
      return 0;
   } else {
      return this->data[r * cols + c];
   }
}
\end{minted}
\end{codebox}
%
\end{frame}

% =========================================================================== %

\begin{frame}[fragile]
%
\begin{columns}[T]
\column{.55\linewidth}
%
\begin{codebox}[... Fortsetzung]
\begin{minted}[fontsize=\scriptsize, linenos, firstnumber=last]{c++}
int main () {
   Mat Empty;
   Mat ThreeByThree(3, 3);

   cout << Empty.getRows() << endl;
   cout << ThreeByThree.getRows() << endl;
   cout << ThreeByThree.getElement(1, 1);

   return 0;
}
\end{minted}
\end{codebox}
%
\begin{itemize}
\item Denkweise: \texttt{Objekt.Aktion(Parameter)}
\end{itemize}
%	
\column{.45\linewidth}
\begin{itemize}
\item Zugriffskontrolle erhöht Anwendungssicherheit
\item Constructors: Objekt automatisch sinnvoll vorbereiten
\item Destructor: Zugriff auf Ressourcen freigeben
\item \mintinline{c++}{new} und \mintinline{c++}{delete}: Wie \mintinline{c++}{malloc} und 
	\mintinline{c++}{free}, aber rufen automatisch Constructors und Destructors auf
\item Overloaded Functions: Selber Name, unterschiedliche Parameterliste, unterschiedlicher Code.
\end{itemize}
\end{columns}
%
\end{frame}

% =========================================================================== %

\begin{frame}[fragile]
%
\begin{columns}[T]
\column{.51\linewidth}
\begin{Large}
{Templates}
\vspace{10pt}
\end{Large}
\begin{itemize}
\item Typen-unabhängige, abstrakte Formulierung von Routinen und Klassen
\item Klasse: Hier jeder beliebige Datentyp, also primitive Typen (z.\,B. \mintinline{c}{int}) bis hin zu
	Klassen wie zuvor kennen gelernt
\item Intern: Umsetzung ähnlich wie Makros: Präprozessor erstellt passenden Code
\item Template-Parameter: Klasse
\item Auch möglich: Template-Parameter Wert
	\begin{itemize}
	\item Erzeugt Funktion, in der Parameter \emph{hardcoded} auftaucht
	\item Manchmal dadurch Performance-Gewinn
	\end{itemize}
\end{itemize}
%
\column{.5\linewidth}
\begin{codebox}[Beispiel: Templates]
\begin{minted}[fontsize=\scriptsize, linenos]{c}
#include <iostream>
using namespace std;

template <class T>
T sum (T a, T b) {
    T result;
    result = a + b;
    return result;
}

int main () {
    int    i=5  , j=6  , k;
    double f=2.0, g=0.5, h;
    k = sum<int>   (i, j);
    h = sum<double>(f, g);
    cout << k << h << endl;
    return 0;
}
\end{minted}
\tiny \url{http://www.cplusplus.com/doc/tutorial/functions2/}
\end{codebox}
\end{columns}
%
\end{frame}

% =========================================================================== %

\begin{frame}{Template-Klasse Vector}
%
\begin{columns}[T]
\column{.5\linewidth}
\begin{itemize}
\item \mintinline{c++}{vector}: \emph{Template-Klasse} aus eigener Library
\item \enquote{n-Tupel gleichen Datentyps} -- Liste von n Werten
\item Interface für C-Arrays
\item Template: Für jeden beliebigen Datentyp machbar -- auch für eigene \texttt{struct}s oder Klassen.
\end{itemize}
%
\column{.5\linewidth}
\begin{itemize}
\item Methoden:
	\begin{itemize}
	\item \texttt{at}, \texttt{size}, \texttt{capacity} -- wie bei \texttt{string}
	\item \texttt{cbegin}, \texttt{cend} -- erstes und letztes Element
	\item \texttt{clear}, \texttt{erase}, \texttt{swap}, \texttt{resize}
	\end{itemize}
\item Bereitgestellte Algorithmen
	\begin{itemize}
	\item \texttt{sort}
	\item \texttt{for\_each} -- Schleifen mit \texttt{vector}s
	\item \texttt{count\_if} -- Zählen mit Bedingung
	\end{itemize}
\item \url{http://en.cppreference.com/w/cpp/container/vector}
\end{itemize}
\end{columns}
%
\end{frame}

% =========================================================================== %

\begin{frame}[fragile]
%
\tcbset{width=.495\linewidth, on line, height=7.8cm}
%
\begin{codebox}[Beispiel: Vectors]
\begin{minted}[fontsize=\scriptsize, linenos]{c}
#include <iostream>
#include <vector>
 
void print_vec(
    const std::vector<int>& vec
) {
    for (auto x: vec) {
         std::cout << ' ' << x;
    }
    std::cout << '\n';
}
 
int main () {
    std::vector<int> vec(3,100);
    print_vec(vec);
 
    auto it = vec.begin();
    it = vec.insert(it, 200);
    print_vec(vec);
\end{minted}
\end{codebox}
%	
\begin{codebox}[Beispiel: ... Fortsetzung]
\begin{minted}[fontsize=\scriptsize, linenos, firstnumber=last]{c}
    vec.insert(it,2,300);
    print_vec(vec);
 
    // it no longer valid, get new one:
    it = vec.begin();
 
    std::vector<int> vec2(2,400);
    vec.insert(
        it+2, 
        vec2.begin(), 
        vec2.end()
    );
    print_vec(vec);
 
    int arr[] = { 501,502,503 };
    vec.insert(vec.begin(), arr, arr+3);
    print_vec(vec);
}
\end{minted}
\tiny \url{http://en.cppreference.com/w/cpp/container/vector/insert}
\end{codebox}
%
\end{frame}