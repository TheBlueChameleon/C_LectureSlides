% =========================================================================== %

\begin{frame}[t,plain]
\titlepage
\end{frame}

% =========================================================================== %

\begin{frame}{Script}
%
\begin{itemize}
\item Kapitel 9
	\begin{itemize}
	\item 9.1. Scopes
	\item 9.2. Funktionen
		\begin{itemize}
		\item 9.2.1. Scopes in bisher bekannten Strukturen
		\item 9.2.2. Forward Declaration
		\item 9.2.3. Übergabe \enquote{By Reference} und \enquote{By Value}
		\item 9.2.4. Datentyp void
		\item 9.2.5. Globale Variablen
		\item 9.2.6. Was als Funktion auslagern
		\end{itemize}
	\end{itemize}
\end{itemize}
%
\end{frame}

% =========================================================================== %

\begin{frame}[fragile]
%
\begin{columns}[T]
\column{.3\linewidth}
\begin{Large}
Funktionen
\vspace{10pt}
\end{Large}
%
\begin{itemize}
\item Häufig auftretende Arbeitsschritte zu Einheit verpacken
\item Von Parametern abhängig gestaltbar
\item Rückgabewert
\item Muss vor Aufruf definiert sein
\item Die Meisten Befehle bisher waren bereits Funktionen
\end{itemize}
%
\column{.7\linewidth}
\begin{codebox}[Syntax: Deklaration (Funktions-Prototyp)]
\begin{minted}[fontsize=\footnotesize]{c}
Rueckgabetyp Funktionsname ( Parameterliste );
\end{minted}
\end{codebox}
%
\begin{codebox}[Syntax: Funktionskörper]
\begin{minted}[fontsize=\footnotesize]{c}
Rueckgabetyp Funktionsname ( Parameterliste ) {
   Anweisungen
   return Rueckgabewert;
}
\end{minted}
\end{codebox}
%
\begin{codebox}[Syntax: Aufruf]
\begin{minted}[fontsize=\footnotesize]{c}
Funktionsname ( Parameterliste )
\end{minted}
\end{codebox}
\end{columns}
%
\end{frame}

% =========================================================================== %

\begin{frame}{Für MathematikerInnen}
%
Funktion: Abbildung aus einem Raum in einen anderen.
\begin{center}

\end{center}
\begin{minipage}{.49\linewidth}
Beispiel: $f: \mathbb{R} \to \mathbb{R}$

Entsprechung in C:\\
\mintinline{c}{double f(double x);}
\end{minipage}
%
\begin{minipage}{.49\linewidth}
Beispiel: $g: \mathbb{R} \times \mathbb{Z} \to \mathbb{R}$

Entsprechung in C:\\
\mintinline{c}{double g(double x, int n);}
\end{minipage}
%
\end{frame}

% =========================================================================== %

\begin{frame}[fragile]
%
\begin{codebox}[Beispiel: Mit Deklaration]
\begin{minted}[fontsize=\scriptsize,linenos]{c}
#include <stdio.h>

double polyNomNomNom(double x);
// ========================================================================= //
int main() {
  printf(   "%lf\n", polyNomNomNom( 1 )   );
  printf(   "%lf\n", polyNomNomNom(2.5)   );
}
// ========================================================================= //
double polyNomNomNom(double x) {
  return 3.7 * x * x * x   +   1.4 * x * x   -   x   +   1;
}
\end{minted}
\end{codebox}
%
\begin{cmdbox}[Ausgabe]
\begin{minted}[fontsize=\scriptsize]{text}
5.100000
65.062500
\end{minted}
\end{cmdbox}
%
\end{frame}

% =========================================================================== %

\begin{frame}[fragile]{Beispiel: Ohne Deklaration}
%
\begin{codebox}
\begin{minted}[fontsize=\scriptsize,linenos]{c}
#include <stdio.h>

double polyNomNomNom(double x) {
  return 3.7 * x * x * x   +   1.4 * x * x   -   x   +   1;
}

// ========================================================================= //

int main() {
  printf(   "%lf\n", polyNomNomNom( 1 )   );
  printf(   "%lf\n", polyNomNomNom(2.5)   );
}
\end{minted}
\end{codebox}
%
\end{frame}

% =========================================================================== %

\begin{frame}[fragile]%{\mintinline{c}{void} -- der Nicht-Datentyp}
%
\begin{columns}[T]
\column{.43\linewidth}
\begin{Large}
\mintinline{c}{void} -- der Nicht-Datentyp
\vspace{10pt}
\end{Large}
\begin{itemize}
\item Funktionen, die kein \emph{Argument} erwarten: \mintinline{c}{void}
\item Aufruf: \mintinline{c}{funcName();}
\item Rückgabetyp; früher auch in Parameterliste
	\begin{itemize}
	\item \mintinline{c}{int main (void)}
	\end{itemize}
\item Verlassen einer \mintinline{c}{void}-Funktion mit \mintinline{c}{return;}
\end{itemize}
%
\column{.5\linewidth}
\begin{codebox}
\begin{minted}[fontsize=\scriptsize,linenos]{c}
#include <stdio.h>

void hieronymus() {
  printf("Lauser\n");
}

int main () {
  hieronymus();
}
\end{minted}
\end{codebox}
\end{columns}
%
\begin{hintbox}
\texttt{funcName} ohne Parameterliste ist nur ein Pointer!
\end{hintbox}
%
\end{frame}

% =========================================================================== %

\begin{frame}[fragile]{Scopes}
%
\begin{columns}[b]
\column{.5\linewidth}
\begin{codebox}
\begin{minted}[fontsize=\scriptsize, linenos]{c}
#include <stdio.h>
int main () {
  int x = 7, z = 2;
  {
    int x = 9, y = 3;
    printf("Inner Scope:\n");
    printf("   x: %d\n", x);
    printf("   y: %d\n", y);
    printf("   z: %d\n", z);
  }
  printf("Outer Scope:\n");
  printf("   x: %d\n", x);
//printf("   y: %d\n", y); undefiniert 
  printf("   z: %d\n", z);
}
\end{minted}
\end{codebox}
%
\column{.5\linewidth}
\begin{itemize}
\item Regeln \enquote{Sichtbarkeit} von Variablen
%\item  \{Umgebungen\} 
\item Gleiche Namen erlaubt -- unterschiedliche Speicherbereiche
\item Sichtbarkeit nur von außen nach innen
\end{itemize}
%
\begin{cmdbox}[Ausgabe]
\begin{minted}[fontsize=\scriptsize]{text}
Inner Scope:
   x: 9
   y: 3
   z: 2
Outer Scope:
   x: 7
   z: 2
\end{minted}
\end{cmdbox}
\end{columns}
%
\end{frame}

% =========================================================================== %

\begin{frame}[fragile]%Scopes in Funktionen
%
\begin{columns}[b]
\column{.5\linewidth}
\begin{Large}
\vspace{10pt}
Scopes in Funktionen
\vspace{6pt}
\end{Large}
\begin{itemize}
\item Funktion definiert eigenen Scope
\item Parameter \enquote{leben} in diesem Scope
\item Lokale Kopien
\item Variablen aus \texttt{main} nicht in \texttt{func} sichtbar
\item \texttt{x} aus \texttt{main} ist nicht dasselbe Objekt wie \texttt{x} in \texttt{func}
\end{itemize}
%
\begin{hintbox}
Benutzt sprechende Namen für\newline Funktionsparameter!
\end{hintbox}
%
\column{.5\linewidth}
\begin{codebox}
\begin{minted}[fontsize=\scriptsize, linenos]{c}
#include <stdio.h>

void func(int x) {
  int y = 3;
  x += y;
  printf("in func: %d\n", x);
}
int main () {
  int x = 3;   
  func(x);   
  printf("in main: %d\n", x);
//printf("%d\n", y); undefiniert
}
\end{minted}
\end{codebox}
%
\begin{cmdbox}[Ausgabe]
\begin{minted}[fontsize=\scriptsize]{text}
in func: 6
in main: 3
\end{minted}
\end{cmdbox}
\end{columns}
%
\end{frame}

% =========================================================================== %

\begin{frame}[fragile]{Lokale Kopien}
%
\begin{itemize}
\item Deklaration einer Funktion: Neue Variable für jeden Parameter
\item Funktionsaufruf: Kopieren eines Werts an die Parameter-Speicherstelle
\item Funktion arbeitet mit der lokalen Kopie
\item Original-Wert wird nicht verändert
\end{itemize}
%
\begin{tcbraster}[raster columns=2, raster equal height, nobeforeafter]
\begin{codebox}
\begin{minted}[linenos, fontsize=\scriptsize]{c}
#include <stdio.h>

void func(int x) {x = 5;}

int main () {
  int x = 2;
  func(x);
  printf("%d\n", x);
}
\end{minted}
\end{codebox}
%
\begin{tcolorbox}[title=T]
%\scriptsize
\begin{tabular}{c|cc}
	Zeile & \texttt{x} in \texttt{main} & \texttt{x} in \texttt{func} \tabcrlf
	6     & 2 & ??  \\
	7     & 2 &  2  \\
	3     & 2 &  5  \\
	8     & 2 & (5) \\
\end{tabular}
\vphantom{h}
\end{tcolorbox}
\end{tcbraster}
%
\end{frame}

% =========================================================================== %

\begin{frame}{Warum Scopes}
%
\begin{itemize}
\item Funktionen: \enquote{abgeschlossene Arbeitspakete}
\item Sollen weitgehend unabhängig vom Rest des Codes funktionieren
\item Parameterliste als \enquote{Eingangsportal}, Rückgabewert als \emph{Ausgangsportal}
\item Sonst: Keine Wechselwirkung
\item Erlaubt, in komplexen Strukturen zu denken, ohne \emph{jede Einzelheit} des Codes im Auge zu
	behalten
\item In Funktionen daher: Lokale Kopien -- minimale Wechselwirkung
\end{itemize}
%
\end{frame}

% =========================================================================== %

\begin{frame}[fragile]
%
\tcbset{width=.495\linewidth, height=7.6cm, on line}
%
\begin{codebox}
\begin{minted}[fontsize=\scriptsize, linenos]{c}
#include <stdio.h>
#include <math.h>

double myExp(double x0, int accuracy) {
  double x = 1.0;
  double denominator = 1.0;
  double result = 1.0;

  for (int i=1; i<accuracy; i++) {
    x           *= x0;
    denominator *= i;
    result      += x / denominator;
  }
  
  return result;
}
\end{minted}
\end{codebox}
%
\begin{codebox}
\begin{minted}[fontsize=\scriptsize, linenos, firstnumber=last]{c}
int main () {
  double d, e;
  
  printf("Genauigkeit\t"
         "Näherung\t"
         "Abweichung\t"
         "Relativ\n"
  );
  
  for (int i=0; i<10; i++) {
    e = myExp(1, i);
    d = e - exp(1);
    printf("%d\t"
           "%+8.6lf\t"
           "%+8.6lf\t"
           "%8.3lf\n",
           i, e, d, d/e
    );
  }
}
\end{minted}
\end{codebox}
%
\end{frame}

% =========================================================================== %

\begin{frame}[fragile]
%
\begin{cmdbox}[Ausführungseispiel: Funktion \texttt{myExp} in Anwendung]
\begin{minted}[fontsize=\scriptsize]{text}
Genauigkeit     Näherung        Abweichung      Relativ
0               +1.000000       -1.718282         -1.718
1               +1.000000       -1.718282         -1.718
2               +2.000000       -0.718282         -0.359
3               +2.500000       -0.218282         -0.087
4               +2.666667       -0.051615         -0.019
5               +2.708333       -0.009948         -0.004
6               +2.716667       -0.001615         -0.001
7               +2.718056       -0.000226         -0.000
8               +2.718254       -0.000028         -0.000
9               +2.718279       -0.000003         -0.000
\end{minted}
\end{cmdbox}
%
\end{frame}

% =========================================================================== %

\begin{frame}[fragile]%{Globale Variablen}
%
\begin{columns}[b]
\column{.45\linewidth}
\begin{Large}
{Globale Variablen}
\vspace{10pt}
\end{Large}
\begin{itemize}
\item Deklaration außerhalb von Functions
\item \enquote{eine Ebene höher} als selbst \mintinline{c}{int main()}
\item Verhalten wie immer bei Scopes
\item Sparsam gebrauchen!
\end{itemize}
%
\begin{cmdbox}[Ausgabe]
\begin{minted}[fontsize=\scriptsize]{text}
main:
  global   : 7
  overwrite: 6
func:
  global   : 7
  overwrite: 8
\end{minted}
\end{cmdbox}
%
\column{.53\linewidth}
\begin{codebox}
\begin{minted}[fontsize=\scriptsize, linenos]{c}
#include <stdio.h>
int global    = 7;
int overwrite = 8;

void func() {
  printf("func:\n");
  printf("  global   : %d\n", global   );
  printf("  overwrite: %d\n", overwrite);
}

int main () {
  int overwrite = 6;
  
  printf("main:\n");
  printf("  global   : %d\n", global   );
  printf("  overwrite: %d\n", overwrite);
  
  func();
}
\end{minted}
\end{codebox}
\end{columns}
%
\end{frame}

% =========================================================================== %

\begin{frame}[fragile]{Werte zurück geben}
%
\tcbset{width=.495\linewidth, height=5cm, on line}
%
\begin{itembox}[Einzelnen Wert]
\item Über \mintinline{c}{return}
\item Muss zum Rückgabetyp passen
\item Verlässt Funktion an dieser Stelle
\item Mehrere \mintinline{c}{return}-Anweisungen möglich (Bedingungen)
\item Trotzdem nur ein Rückgabewert
\end{itembox}
%
\begin{itembox}[mehrere Werte]
\item Übergabe von Pointern
\item Funktion schreibt an Adresse von Pointern
\item Vergleiche: \texttt{scanf}
\end{itembox}
%
\end{frame}

% =========================================================================== %

\begin{frame}[fragile]%{Wertrückgabe mit \mintinline{c}{return}}
%
\tcbset{width=.6\linewidth, height=7.5cm, on line}
\begin{codebox}
\begin{minted}[fontsize=\scriptsize, linenos]{c}
#include <stdio.h>
#include <math.h>

double compute(
  double x, 
  char   op, 
  double y) 
{
  switch (op) {
    case '+': return x + y;
    case '-': return x - y;
    case '*': return x * y;
    case '/':
      if (!y) {printf("division by zero\n");}
      return x / y;            // return #inf
  }
  
  printf ("illegal operator\n");
  return NAN;
}
\end{minted}
\end{codebox}
%
\tcbset{width=.38\linewidth, on line}
\begin{codebox}[... Fortsetzung]
\begin{minted}[fontsize=\scriptsize, linenos, firstnumber=last]{c}
int main () {
  double a, b;
  char c;

  printf("Number 1:\n");
  scanf("%lf", &a);

  printf("Operator:\n");
  scanf(" %c", &c);

  printf("Number 2:\n");
  scanf("%lf", &b);

  printf(
    "%lf %c %lf = %lf",
    a, c, b,
    compute(a, c, b)
  );
}
\end{minted}
\end{codebox}
%
\end{frame}

% =========================================================================== %

\begin{frame}[fragile]{Wertübergabe mit Pointern als Parameter}
%
\tcbset{width=.495\linewidth, on line}
%
\begin{codebox}[Code, equal height group=gByRef]
\begin{minted}[fontsize=\scriptsize, linenos]{c}
#include <stdio.h>

void func(int x, int * y) {
  printf("in func:\n");
  printf("   before change: ");
  printf("x: %d; y: %d\n", x, *y);

   x = 3;
  *y = 3;

  printf("   after  change: ");
  printf("x: %d; y: %d\n", x, *y);
}
\end{minted}
\end{codebox}
%
\begin{codebox}[... Fortsetzung, equal height group=gByRef]
\begin{minted}[fontsize=\scriptsize, linenos, firstnumber=last]{c}
int main () {
  int a = 1, b = 1;

  printf("in main:\n");
  printf("   before call  : ");
  printf("a: %d; b: %d\n", a, b);

  func(a, &b);

  printf("in main:\n");
  printf("   after  call  : ");
  printf("a: %d; b: %d\n", a, b);
}
\end{minted}
\end{codebox}
%
\end{frame}

% =========================================================================== %

\begin{frame}{Was als Funktion \enquote{auslagern}?}
%
\begin{columns}
\column{.5\linewidth}
\begin{itemize}
\item Häufig wiederkehrende Berechnungen\newline
	Beispiel: Skalarprodukt
\item Insbesondere: Parametrisierte Probleme\newline
	Beispiel: Bestimme Anzahl eines bestimmten Buchstaben in einem Text
\item Übersichtliche Gestaltung von Code (sprechende Namen!)\newline
	Beispiel \texttt{compute(a, operator, b)}
\end{itemize}
%
\column{.5\linewidth}
\begin{itemize}
\item Speziallösungen für Probleme, zu denen nur eine allgemeine Lösung zur Verfügung steht
\item Vorbereiten von Objekten, mit denen weiter gerechnet wird
\item[$\Rightarrow$] Erst durch Funktionen sind komplexe Programme möglich!
\end{itemize}
\end{columns}
%
\begin{hintbox}
Funktionen können auch andere Funktionen aufrufen.
\end{hintbox}
%
\end{frame}

% =========================================================================== %