% =========================================================================== %

\begin{frame}[t,plain]
\titlepage
\end{frame}

% =========================================================================== %

\begin{frame}{Script}
%
\begin{itemize}
\item Kapitel 9
	\begin{itemize}
	\item 9.2.8. Speicherklasse \inC{static}
	\end{itemize}
\item Kapitel 11
	\begin{itemize}
	\item 11.1. Trennung von Header- und Modul-Code
	\item 11.2. Speicherklasse \inC{extern}
	\item 11.3. Funktionen mit eingeschränkter Sichtbarkeit: \inC{static}
	\end{itemize}
\end{itemize}
%
\end{frame}

% =========================================================================== %

\begin{frame}[fragile]{Modifier}
%
\begin{itemize}
\item Deklaration: \texttt{Datentyp Variablenname}
\item Zusatz-Informationen möglich
\item Modifier \mintinline{c}{unsigned, const, restrict, volatile}
	\begin{itemize}
	\item verändern das Laufzeitverhalten von Variablen
	\item Optimierung durch Compiler
	\end{itemize}
\item Speicherklassen \mintinline{c}{static, extern, register}
	\begin{itemize}
	\item Steuern Lebensdauer und Sichtbarkeit
	\end{itemize}
\item[$\Rightarrow$] \texttt{Speicherklasse Modifier Datentyp Variablenname}
	\begin{itemize}
	\item Reihenfolge tatsächlich beliebig
	\end{itemize}
\end{itemize}
%
\end{frame}

% =========================================================================== %

\begin{frame}[fragile]{\mintinline{c}{unsigned}}
%
\begin{columns}[T]
\column{.5\linewidth}
\begin{itemize}
\item Abwandlung Ganzzahl-Datentypen
	\begin{itemize}
	\item \mintinline{c}{char}, \mintinline{c}{short}, \mintinline{c}{int}, \mintinline{c}{long},
		  \mintinline{c}{long long}
	\end{itemize}
\item Erlaubt nur positive Werte
\item Dafür doppelter Betrag
\item Unintuitives Verhalten: \emph{Buffer underflow}
\item Ausgabe mit \texttt{printf}: Modifier Formatzeichen \texttt{u}
	\begin{itemize}
	\item Weitere Typen: \texttt{hh} für \mintinline{c}{char}
	\item Weitere Typen: \texttt{h} für \mintinline{c}{short}
	\item Weitere Typen: \texttt{d} für \mintinline{c}{int}
	\item Weitere Typen: \texttt{l} für \mintinline{c}{long}
	\item Weitere Typen: \texttt{ll} für \mintinline{c}{long long}
	\end{itemize}
\end{itemize}
%
\column{.5\linewidth}
\vspace{-10pt}
\begin{codebox}
\begin{minted}[fontsize=\scriptsize, linenos]{c}
#include <stdio.h>

int main () {
  unsigned char x = 17; // binär: 00010001
  printf("%hhu\n", x);
  
  x -= 18;              // -> -1: 11111111
  printf("%hhu\n", x);
}
\end{minted}
\end{codebox}
%
\begin{cmdbox}[Ausgabe]
\scriptsize
17\\
255
\end{cmdbox}
\end{columns}
%
\end{frame}

% =========================================================================== %

\begin{frame}[fragile]{\inC{const}}
%
\begin{columns}[T]
\column{.5\linewidth}
\begin{itemize}
\item Verhindert Wert-Änderung nach Initialisierung
\item Sicherheitsmechanismus
\item Häufig für Naturkonstanten (globale \enquote{Variablen})
\item Manchmal auch in Funktions-Signaturen
	\begin{itemize}
	\item CPP-Referenz
	\end{itemize}
\item Kann mit \mintinline{c}{unsigned} kombiniert werden
\end{itemize}
%
\column{.5\linewidth}
\vspace{-10pt}
\begin{codebox}
\begin{minted}[fontsize=\scriptsize, linenos]{c}
#include <stdio.h>

const double PI = 3.141592654;

double circleArea(double r) {
  return r * r * PI;
}

int main () {
  printf("Kreisfläche: %lf\n", 
         circleArea(80085)
        );
  
  // PI = 3.0; -- Verboten
}
\end{minted}
\end{codebox}
\end{columns}
%
\end{frame}

% =========================================================================== %

\begin{frame}[fragile]{\mintinline{c}{restrict}, \mintinline{c}{volatile} und \mintinline{c}{register}}
%
\begin{columns}[T]
\column{.5\linewidth}
\begin{itemize}
\item \mintinline{c}{restrict}: Optimierung Pointer-Zugriffe
\item Kein \enquote{Überlappen} referenzierter Speicherbereiche
\item Beeinflusst interne Umsetzung durch Compiler
\end{itemize}

\vspace{6pt}
\begin{itemize}
\item \mintinline{c}{volatile}: Hinweis, dass Variable von anderen Prozessen geändert werden kann
\item Verhindert Optimierungsschritte des Compilers
\end{itemize}
%
\column{.5\linewidth}
\begin{itemize}
\item \mintinline{c}{register}: Ablegen im \enquote{Prozessor-Speicher}
\item Optimierung für Variablen, die besonders häufig gelesen/geändert werden
\item Wird mit Standard C 2020 nicht mehr verwendet
\item Weiterhin reserviertes Schlüsselwort
\end{itemize}
\end{columns}
%
\end{frame}

% =========================================================================== %

\begin{frame}[fragile]{Speicherklasse \inC{static}}
%
\begin{columns}[T]
\column{.5\linewidth}
\begin{itemize}
\item \enquote{Residente Variablen}
\item Speicherstelle wird nicht frei gegeben, wenn Scope verlassen wird
\item Optionaler Startwert wird nur ein einziges Mal zugewiesen
\item Symbol dennoch nur auf Scope begrenzt
\item Nützlich in Funktionen
\item Häufig: \enquote{setze Suche fort}
\item Ausblick: Technik \emph{Rekursion}
\end{itemize}
%
\column{.5\linewidth}
\vspace{-10pt}
\begin{codebox}
\begin{minted}[fontsize=\scriptsize, linenos]{c}
void func () {
  static int calls = 0;
  printf("Aufruf %d von func\n", ++calls);
}

int main () {
  for (int i=0; i<3; ++i) {func();}
}
\end{minted}
\end{codebox}
%
\begin{cmdbox}[Ausgabe]
\scriptsize
Aufruf 1 von func\\
Aufruf 2 von func\\
Aufruf 3 von func
\end{cmdbox}
\end{columns}
%
\end{frame}

% =========================================================================== %

\begin{frame}[fragile]{Modularisierung -- Motivation}
%
\begin{itemize}
\item Erinnerung: Mathe-Bibliothek
	\begin{itemize}
	\item Braucht \mintinline{c}{#include <math.h>}
	\item Braucht Compiler-Option \texttt{-lm}
	\end{itemize}
\item Erlaubt, alte Funktionen wieder zu verwenden, ohne sie extra in den Code zu kopieren
	\begin{itemize}
	\item Bessere Gliederung, höhere Wiederverwendbarkeit
	\item Großprojekte kompilieren schneller
	\item Kommerzielle \enquote{Blackbox-Lösungen}
	\end{itemize}
\end{itemize}

\vspace{10pt}
Unser Ziel hier ist, ebenfalls Bibliotheken zu schreiben.
%
\end{frame}

% =========================================================================== %

\begin{frame}[fragile]{Modularisierung -- Recap \texttt{\#include}}
%
\begin{itemize}
\item \enquote{Copy \& Paste} einer Datei direkt in den Code
\item \mintinline{c}{#include "filename.ext"} bzw. \mintinline{c}{#include <filename.ext>}
\item \texttt{''Double Quotes}: Lade aus aktuellem Verzeichnis
\item \texttt{<angle brackets>}: Lade aus Standard-Verzeichnis
\item Konvention:
	\begin{itemize}
	\item Dateiendung \texttt{.c}: Tatsächlicher Modul-Code -- Funktionen, Anweisungen
	\item Dateiendung \texttt{.h}: \enquote{\emph{Header}} -- Definitionen
	\end{itemize}
\item[\Thus] Übersetzung von jedem Modul als eigenständige Einheit
\item[\Thus] Header enthält nur \enquote{überall} gültige Definitionen
\end{itemize}
%
\end{frame}

% =========================================================================== %

\begin{frame}[fragile]{Inhalt Header}
%
\begin{itemize}
\item Header: \texttt{.h}-Datei
\item Nur \emph{Definitionen}, keine \emph{Deklarationen}
	\begin{itemize}
	\item Funktions-Prototypen
	\item \mintinline{c}{typedef}s
	\item \mintinline{c}{struct}s
	\item \mintinline{c}{union}s
	\item \mintinline{c}{enum}s
	\item[$\Rightarrow$] Alles, was selbst keinen Speicherplatz braucht
	\end{itemize}
\item Werden also in mehreren \texttt{.c}-Dateien via \mintinline{c}{#include "header.h"} eingebunden
\item Dürfen daher keine Deklarationen enthalten, da sonst gedoppelt
\item[\Thus] \enquote{verbinden} mehrere Module
\end{itemize}
%
\end{frame}

% =========================================================================== %

\begin{frame}[fragile]{Module}
%
\begin{columns}[T]
\column{.5\linewidth}
%
\begin{itemize}
\item \enquote{normaler} C-Code
\item Pro Projekt: nur ein Modul mit \texttt{int main()}
\item Meist: Sammlung von Funktionen
\item Kann aber auch globale Variaben enthalten
\item Modul als \enquote{umfassender Scope}
\item[$\Rightarrow$] \enquote{eigene globale Variablen je Modul}
\item Alles, das über die Modulgrenzen sichtbar sein soll, wird im Header definiert
\end{itemize}
%
\column{.5\linewidth}
\begin{cmdbox}
\scriptsize
gcc modul1.c modul2.c ... [Optionen]
\end{cmdbox}
%
\begin{itemize}
\item Nur Module, keine Header
\item Bedenke: Bereits durch \mintinline{c}{#include}-Zeile teil der \texttt{.c}-Datei
\item[\Thus] Erzeugt eine ausführbare Datei, die aus den einzelnen Modulen zusammen gesetzt wurde
\end{itemize}
\end{columns}
%
\end{frame}

% =========================================================================== %

\begin{frame}[fragile]
%
\vspace{-5pt}
\begin{codebox}[Beispiel: Header \texttt{vector2D.h}]
\begin{minted}[linenos, fontsize=\scriptsize]{c}
typedef struct {
  double x;
  double y;
} vector2D;

vector2D v2_sum     (vector2D a, vector2D b);
double   v2_length  (vector2D v);
\end{minted}
\end{codebox}
%
\begin{codebox}[Beispiel: Modulcode \texttt{vector2D.c}]
\begin{minted}[linenos, fontsize=\scriptsize]{c}
#include "vector2D.h"
#include <math.h>

vector2D v2_sum (vector2D a, vector2D b) {
  vector2D reVal = {a.x + b.x, a.y + b.y}; 
  return reVal;
}

double v2_length (vector2D v) {return hypot(v.x, v.y);}
\end{minted}
\end{codebox}
%
\end{frame}

% =========================================================================== %

\begin{frame}[fragile]
%
\adjustbox{valign=t}{\begin{minipage}{.5\linewidth}
\begin{codebox}[Beispiel: Hauptmodulcode \texttt{myProg.c}]
\begin{minted}[linenos, fontsize=\scriptsize]{c}
#include <stdio.h>
#include "vector2D.h"

int main () {
  vector2D v = {1,1}, w = {2, 3};
  
  printf("%lf\n", 
          v2_length(v2_sum(v, w))
        );
}
\end{minted}
\end{codebox}
\end{minipage}}
%
\adjustbox{valign=t}{\begin{minipage}{.45\linewidth}
\begin{cmdbox}
\scriptsize
gcc std=c11 myProg.c vector2D.c -lm -Wall -Wextra
\end{cmdbox}
%
\begin{cmdbox}[Ausgabe]
\scriptsize
5.000000
\end{cmdbox}
\end{minipage}}
%
\begin{itemize}
\item Kein \mintinline{c}{#include <math.h>} im Hauptmodul nötig (aber erlaubt)
\item Aber: \texttt{-lm} beim Compiler-Aufruf nötig
\end{itemize}
%
\end{frame}

% =========================================================================== %

\begin{frame}{Vorkompilierte Bibliotheken}
%
\begin{itemize}
\item Bisher:
	\begin{itemize}
	\item Gliederung in Module, aber
	\item Immer noch jedes Mal jedes Modul komplett übersetzt
	\end{itemize}
\item Anderer Compiler-Aufruf
	\begin{itemize}
	\item 
\begin{cmdbox}[Object-Files erstellen (Zwischenzustand beim Kompilieren)]
\scriptsize
gcc std=c11 myLibrary.c -Wall -Wextra -Wedantic {\color{cyan} -c -fPIC} -o myLibrary.o
\end{cmdbox}
	\item 
\begin{cmdbox}[Zu einer \emph{static library} linken]
\scriptsize
ar rcs myLibrary.a myLibrary.o
\end{cmdbox}
	\end{itemize}
\item Library kann auch mehrere \texttt{.o}-Dateien enthalten
\end{itemize}
%
\end{frame}

% =========================================================================== %

\begin{frame}[fragile]{Library benutzen}
%
\begin{itemize}
\item Normalen \texttt{main}-Code schreiben
\item Darf natürlich auch eigene Funktionen enthalten (vgl: Mathe-Library verbietet das auch nicht)
\item Library-Header mit \mintinline{c}{#include} einbinden
\item Kompilieren mit \texttt{-lMyLibName}
\end{itemize}
%
\begin{cmdbox}[Eigene Library benutzen]
\scriptsize
gcc std=c11 -Wall -Wextra -Wedantic mainMoudle.c {\color{cyan} myLibrary.a} -o myExecutable
\end{cmdbox}
%
\end{frame}

% =========================================================================== %

\begin{frame}[fragile]{Über Module geteilte globale Variablen: Speicherklasse \mintinline{c}{extern}}
%
\begin{columns}[T]
\column{.5\linewidth}
\begin{itemize}
\item Globale Variablen auf Module beschränkt
	\begin{itemize}
	\item Gut so, aber manchmal wollen wir diese Globals \emph{projektweit} nutzen
	\item In Header packen: Copy \& Paste 
	\item[\Thus] Mehrfaches Anlegen
	\end{itemize}
\item Stattdessen: \inC{extern}
	\begin{itemize}
	\item Im Header: \enquote{Sichtbarmachen} einer Variablen
	\item Selbst keine Deklaration, nur Definition
	\item Keine Wertzuweisung in \mintinline{c}{extern}-Zeile
	\end{itemize}
\end{itemize}
%
\column{.5\linewidth}
\begin{codebox}[Syntax]
\inC{extern Datentyp Variable;}
\end{codebox}
\end{columns}
%
\end{frame}

% =========================================================================== %

\begin{frame}[fragile]
%
\begin{codebox}[Beispiel: Hauptmodulcode \texttt{myProg.c}]
\begin{minted}[linenos, fontsize=\scriptsize]{c}
#include <stdio.h>
#include "definitions.h"

int main () {
  printf("Globale Variable 'global' = %lf\n", global);
}
\end{minted}
\end{codebox}

\begin{codebox}[Beispiel: Header \texttt{definitions.h}]
\begin{minted}[linenos, fontsize=\scriptsize]{c}
extern double global;
\end{minted}
\end{codebox}

\begin{codebox}[Beispiel: Modul \texttt{module.c}]
\begin{minted}[linenos, fontsize=\scriptsize]{c}
#include "definitions.h"

double global = 5.0;
\end{minted}
\end{codebox}
%
\end{frame}