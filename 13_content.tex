% =========================================================================== %

\begin{frame}[t,plain]
\titlepage
\end{frame}

% =========================================================================== %

\begin{frame}[fragile]{Recap}
%
\begin{itemize}
\item Function Pointers
	\begin{itemize}
	\item Deklaration:	\tabto{4cm}	\texttt{returntype (*funcname) (parameterlist)}
	\item Aufruf	:		\tabto{4cm}	\texttt{funcname(parameterlist)}
	\item Zuweisung:		\tabto{4cm}	\texttt{funcname = otherfunction}
	\end{itemize}
\item Kommandozeilen-Parameter
	\begin{itemize}
	\item Kommunikation mit Betriebssystem
	\item \mintinline{c}{int main(int argc, char ** argv)}
	\item Aufruf über \texttt{./myCode myParams}
	\end{itemize}
\item Präprozessor
	\begin{itemize}
	\item \mintinline{c}{#define}
	\item \mintinline{c}{#if}, \mintinline{c}{#else if}, \mintinline{c}{#endif}
	\item Makros
	\end{itemize}
\item Fragen?
\end{itemize}
%
\end{frame}

% =========================================================================== %

\begin{frame}[fragile]
%
\begin{columns}[T]
\column{.5\linewidth}
\begin{Large}
Dateiverwaltung
\vspace{10pt}
\end{Large}
%
\begin{itemize}
\item Zugriff über \emph{handle}
	\item Komplexes Datenobjekt, von \texttt{stdio.h} bereits vorbereitet
	\item Pointer auf \texttt{struct} vom Typ \texttt{FILE}
	\item Enthält Dateiname, Pfad, Cursorposition, Dateimodus, ...
\item Handle erhalten: Über \texttt{fopen}
\item Muss wieder geschlossen werden mit \texttt{fclose}
\end{itemize}
%
\column{.5\linewidth}
\begin{codebox}[Syntax: fopen]
\begin{minted}[fontsize=\scriptsize]{c}
FILE * fp = fopen(filename, mode);
...
fclose(fp);
\end{minted}
\end{codebox}
%
\begin{itemize}
\item \texttt{filename}: String -- \mintinline{c}{char *}
\item \texttt{mode}: einer von diesen:
	\begin{itemize}
	\item \texttt{''w''} -- write\newline
		Vorher bestehender Dateiinhalt wird gelöscht
	\item \texttt{''a''} -- append\newline
		Schreibe an das Dateiende
	\item \texttt{''r''} -- read
	\end{itemize}
\end{itemize}
%
\end{columns}
%
\end{frame}

% =========================================================================== %

\begin{frame}[fragile]
%
\begin{columns}[T]
\column{.49\linewidth}
\begin{Large}
In Dateien schreiben
\vspace{10pt}
\end{Large}
%
\begin{itemize}
\item Einfachste Variante: \texttt{fprintf}
\item Wie \texttt{printf}, zusätzlicher Parameter \texttt{handle}
\item Definiert in \texttt{stdio.h}
\item Alle Formattierzeichen, wie von \texttt{printf} bekannt
\item Rückgabewert: Zahl geschriebener Zeichen
\end{itemize}
%
\column{.51\linewidth}
\begin{codebox}[Syntax]
\begin{minted}[fontsize=\footnotesize]{c}
fprintf(handle, formatstr, arglist);
\end{minted}
\end{codebox}
%
\begin{codebox}[Beispiel]
\begin{minted}[fontsize=\scriptsize, linenos]{c}
int a = 7;
FILE * fp = fopen ("myFile.txt", "w");
  fprintf(fp, "literal text\n");
  fprintf(fp, "%d\t%s\n", a, "more text");
fclose(fp);
\end{minted}
\end{codebox}
\end{columns}
%
\end{frame}

% =========================================================================== %

\begin{frame}[fragile]
%
\begin{columns}[T]
\column{.5\linewidth}
\begin{Large}
Arrays schreiben
\vspace{10pt}
\end{Large}
%
\begin{itemize}
\item \emph{Binäres} Schreiben mit \texttt{fwrite}
\begin{itemize}
	\item Zahlen werden nicht als menschenlesbarer Text, sondern als Bitmuster abgelegt
	\item Also Dateiinhalt wie im Speicher
\end{itemize}
\item Definiert in \texttt{stdio.h}
\item Rückgabewert: Zahl geschriebenen Objekte
\item Schneller als \texttt{fprintf}, da keine Übersetzung in menschenlesbares Format nötig.
\end{itemize}
%
\column{.5\linewidth}
\begin{codebox}[Syntax]
\begin{minted}[fontsize=\footnotesize]{c}
fwrite(source, count, size, handle);
\end{minted}
\end{codebox}
%
\begin{itemize}
\item \texttt{source}: Pointer auf Array, das geschrieben werden soll
\item \texttt{count}: Anzahl der zu schreibenden Elemente
\item \texttt{size}: Größe eines Elements
\item \texttt{handle}: Handle zur Datei wie von \texttt{fopen} erhalten
\end{itemize}
\end{columns}
%
\end{frame}

% =========================================================================== %

\begin{frame}[fragile]
%
\tcbset{width=.495\linewidth, on line, height=7.5cm}
%
\begin{codebox}[Beispiel: \texttt{fwrite} und \texttt{fprintf}]
\begin{minted}[fontsize=\scriptsize, linenos]{c}
#include <stdio.h>

int main (void) {
  char alphabet[26];
  int  somenum = 67;
  int i;
  
  for (i=0; i<sizeof(alphabet); i++) {
    alphabet[i] = 'A' + i;
  }
  
  FILE * fp = fopen("file.txt", "w");    
  fprintf(fp, "'alphabet' as "
              "decimals, fprintf:\n");
  for (i=0; i<sizeof(alphabet); i++) {
    fprintf(fp, "%d\t", alphabet[i]);
  }
  fprintf(fp, "|\n\n");
\end{minted}
\end{codebox}
%
\begin{codebox}[...Fortsetzung]
\begin{minted}[fontsize=\scriptsize, linenos, firstnumber=last]{c}
  fprintf(fp, 
    "'alphabet' using fwrite:\n");
  fwrite(alphabet, 26, sizeof(char), fp);
  fprintf(fp, "|\n\n");
  
  fprintf(fp, "'somenum' as "
              "character, fprintf:\n");
  fprintf(fp, "%c", somenum);
  fprintf(fp, "|\n\n");
    
  fprintf(fp, "'somenum' "
              "using fwrite:\n");
  fwrite(&somenum,  1, sizeof(int), fp);
  fprintf(fp, "|\n\n");

  fclose(fp);
}
\end{minted}
\end{codebox}
%
\end{frame}

% =========================================================================== %

\begin{frame}[fragile]
%
\begin{cmdbox}[Dateiausgabe]
\begin{minted}[fontsize=\scriptsize]{text}
Output of 'alphabet' as decimal numbers, using fprintf:
65  66  67  68  69  70  71  72  73  74  75  76  77  78  79  80  81  82  83  84  85  86
  87  88  89  90  |

Output of 'alphabet' using fwrite:
ABCDEFGHIJKLMNOPQRSTUVWXYZ|

Output of 'somenum' as character, using fprintf:
C|

Output of 'somenum' using fwrite:
C   |
\end{minted}
\end{cmdbox}
%
\begin{hintbox}
Beachte die drei \texttt{NULL}-chars bei der Ausgabe mit \texttt{fwrite}
\end{hintbox}
%
\end{frame}

% =========================================================================== %

\begin{frame}[fragile]
%
\begin{columns}[T]
\column{.5\linewidth}
\begin{Large}
Aus Dateien lesen
\vspace{10pt}
\end{Large}
%
\begin{itemize}
\item \texttt{fscanf}: Wie \texttt{scanf}, zusätzlicher Parameter \texttt{handle}
\item Definiert in \texttt{stdio.h}
\item Alle Formattierzeichen, wie von \texttt{scanf} bekannt
\item Datei muss im \emph{Lesemodus} geöffnet sein!
\item Rückgabewert: Zahl erfolgreich gelesener Arumente
\end{itemize}
%
\column{.5\linewidth}
\begin{codebox}[Syntax]
\begin{minted}[fontsize=\footnotesize]{c}
fscanf(handle, formatstr, arglist);
\end{minted}
\end{codebox}
%
\begin{codebox}[Beispiel]
\begin{minted}[fontsize=\scriptsize, linenos]{c}
int a = 7;
FILE * fp = fopen ("myFile.txt", "r");
  int    x;
  double y;
  char   z;
  char string[100];
  
  fscanf(fp, "%d %lf %c", &x, &y, &z);
  fscanf(fp, "%99s", string);
fclose(fp);
\end{minted}
\end{codebox}
\end{columns}
%
\end{frame}

% =========================================================================== %

\begin{frame}[fragile]
%
\begin{columns}[T]
\column{.5\linewidth}
\begin{Large}
Arrays lesen
\vspace{10pt}
\end{Large}
%
\begin{itemize}
\item \emph{Binäres} Lesen mit \texttt{fread}
\begin{itemize}
	\item Gegenoperation zu \texttt{fwrite}
	\item Datei direkt in Speicher laden
	\item Kein Interpretieren der Daten
\end{itemize}
\item Definiert in \texttt{stdio.h}
\item Rückgabewert: Zahl erfolgreich gelesener Objekte
\end{itemize}
%
\column{.5\linewidth}
\begin{codebox}[Syntax]
\begin{minted}[fontsize=\footnotesize]{c}
fread(dest, count, size, handle);
\end{minted}
\end{codebox}
%
\begin{itemize}
\item \texttt{dest}: Pointer auf Array, in das geschrieben werden soll
\item \texttt{count}: Anzahl der zu lesenden Elemente
\item \texttt{size}: Größe eines Elements
\item \texttt{handle}: Handle zur Datei wie von \texttt{fopen} erhalten
\end{itemize}
\end{columns}
%
\end{frame}

% =========================================================================== %

\begin{frame}[fragile]
%
\begin{cmdbox}[Quelldatei]
\begin{minted}[fontsize=\scriptsize]{text}
Output of 'alphabet' as decimal numbers, using fprintf:
65  66  67  68  69  70  71  72  73  74  75  76  77  78  79  80  81  82  83  84  85  86
  87  88  89  90  |

Output of 'alphabet' using fwrite:
ABCDEFGHIJKLMNOPQRSTUVWXYZ|

Output of 'somenum' as character, using fprintf:
C|

Output of 'somenum' using fwrite:
C   |
\end{minted}
\end{cmdbox}
%
\end{frame}

% =========================================================================== %

\begin{frame}[fragile]
%
\tcbset{width=.495\linewidth, on line, height=7.2cm}
%
\begin{codebox}[Beispiel: \texttt{fscanf} und \texttt{fread}]
\begin{minted}[fontsize=\scriptsize, linenos]{c}
#include <stdio.h>

void gotoEnter(FILE * fp) {
  char current = 0;
  
  do {
    fread(&current, 
        1, sizeof(char), 
        fp);
  } while (current != '\n');
}

int main (void) {
  char alphabet[26] = {};
  int  somenum, i;
  
  FILE * fp = fopen("file.txt", "r");
    gotoEnter(fp);
\end{minted}
\end{codebox}
%
\begin{codebox}[... Fortsetzung ...]
\begin{minted}[fontsize=\scriptsize, linenos, firstnumber=last]{c}
    printf("fscanf:\t");
    for (i=0; i<26; i++) {
      fscanf(fp, "%hhi", alphabet + i);
    }
    
    for (i=0; i<26; i++) {
      printf("%c ", alphabet[i]);
    }
    printf("\n");
    
    for(i=0; i<3; i++) {gotoEnter(fp);}
    
    printf("fread:\t");
    fread(alphabet, 26, 1, fp);
    for (i=0; i<26; i++) {
      printf("%c ", alphabet[i]);
    }
    printf("\n");
\end{minted}
\end{codebox}
%
\end{frame}

% =========================================================================== %

\begin{frame}[fragile]
%
\begin{codebox}[... Fortsetzung]
\begin{minted}[fontsize=\scriptsize, linenos, firstnumber=last]{c}
    for(i=0; i<3; i++) {gotoEnter(fp);}
    
    fscanf(fp, "%c", (char *) (&somenum));
    printf("fscanf:\t%d\n", somenum);
    
    for(i=0; i<3; i++) {gotoEnter(fp);}
    
    fread(&somenum, 1, sizeof(int), fp);
    printf("fread:\t%d\n", somenum);
  fclose(fp);
}
\end{minted}
\end{codebox}
%
\begin{cmdbox}[Ausgabe]
\begin{minted}[fontsize=\scriptsize]{text}
fscanf: A B C D E F G H I J K L M N O P Q R S T U V W X Y Z 
fread:  A B C D E F G H I J K L M N O P Q R S T U V W X Y Z 
fscanf: 67
fread:  67
\end{minted}
\end{cmdbox}
%
\end{frame}

% =========================================================================== %

\begin{frame}{Wahl der Lese- und Schreibmethode}
%
\tcbset{width=.495\linewidth, on line, height=3.5cm}
%
\begin{itembox}[\texttt{fread} und \texttt{fwrite}]
\item Dateien, die nur für den Computer bestimmt sind
	\begin{itemize}
	\item Bilder, Audio, Video
	\item Alles in Arrays
	\end{itemize}
\item Ausschnitte von \mintinline{c}{char}-Arrays
\end{itembox}
%
\begin{itembox}[\texttt{fprintf} und \texttt{fscanf}]
\item Alles, das für den Menschen lesbar sein soll
\end{itembox}
%
\begin{tiny}
\\
\end{tiny}
%
\tcbset{width=\linewidth, height=2cm}
%
\begin{hintbox}
Lese- und Schreibmethode aufeinander abstimmen: \texttt{fread} mit \texttt{fwrite} oder \texttt{fprintf} mit \texttt{fscanf}
\end{hintbox}
%
\end{frame}

% =========================================================================== %

\begin{frame}[fragile]{Cursorposition in Datei bestimmen und ändern: \texttt{ftell} und \texttt{fseek}}
%
\begin{columns}[T]
\column{.5\linewidth}
\begin{codebox}[Syntax]
\begin{minted}[fontsize=\scriptsize]{c}
long int offset = ftell (pFile);
fseek (pFile, offset, origin);
\end{minted}
\end{codebox}
%
\begin{itemize}
\item \texttt{ftell}: Bytes ab Dateianfang
\item \texttt{fseek}: Setze Cursor in Datei
	\begin{itemize}
	\item \texttt{pFile}: Handle zur Datei
	\item \texttt{offset}: Neue Cursorposition in Bytes, bezogen auf Referenzpunkt
	\item \texttt{origin}: Referenzpunkt:
		\begin{itemize}
		\item \texttt{SEEK\_SET} -- Dateianfang
		\item \texttt{SEEK\_CUR} -- Aktuelle Cursorposition
		\item \texttt{SEEK\_END} -- Dateiende
		\end{itemize}
	\end{itemize}
\end{itemize}
%
\column{.5\linewidth}
\begin{codebox}[Beispiel: \texttt{fseek}]
\begin{minted}[fontsize=\scriptsize, linenos]{c}
#include <stdio.h>

int main ()
{
  FILE * pFile;
  pFile = fopen ( "example.txt" , "w" );
  fputs ( "This is an apple." , pFile );
  fseek ( pFile , 9 , SEEK_SET );
  fputs ( " sam" , pFile );
  fclose ( pFile );
  return 0;
}
\end{minted}
\tiny Quelle:\newline
	\url{http://www.cplusplus.com/reference/cstdio/ftell/}
\end{codebox}
\end{columns}
%
\end{frame}

% =========================================================================== %

\begin{frame}[fragile]
%
\begin{columns}[T]
\column{.5\linewidth}
\begin{Large}
Umgang mit Dateiende -- \texttt{feof}
\vspace{10pt}
\end{Large}
%
\begin{codebox}[Syntax]
\begin{minted}[fontsize=\scriptsize, linenos]{c}
endReached = feof(handle);
\end{minted}
\end{codebox}
%
\begin{itemize}
\item Lese-Befehle hinter Dateiende erzeugen Fehler
\item Datei-Ende erkennen mit \texttt{feof}
\item Rückgabe: Wahrheitswert
	\begin{itemize}
	\item \enquote{nonzero} wenn Dateiende erreicht
	\item ansonsten 0.
	\end{itemize}
\end{itemize}
%
\column{.5\linewidth}
\begin{codebox}[Beispiel: \texttt{feof}]
\begin{minted}[fontsize=\scriptsize, linenos]{c}
#include <stdio.h>

int main () {
  char string[1024] = {};
  int i = 0;
  
  FILE * pFile = fopen ("file.dat", "r");
  do {
     fread(string + i, 1, 1, pFile);
     if (++i == 1024) {break;}
  } while (!feof(pFile));

  printf("%s\n", string);  
  
  fclose(pFile);
}
\end{minted}
\end{codebox}
\end{columns}
%
\end{frame}

% =========================================================================== %

\begin{frame}[fragile]
%
\begin{columns}[T]
\column{.5\linewidth}
\begin{Large}
Dateilänge bestimmen
\vspace{10pt}
\end{Large}
%
\begin{itemize}
\item Kein eigenständiger Befehl
\item Ausnutzung von \texttt{fseek} und \texttt{ftell}
\item Springe ans Ende
\item bestimme Position 
\item Springe zurück
\end{itemize}
%
\column{.5\linewidth}
\begin{codebox}[Beispiel: Dateilänge bestimmen]
\begin{minted}[fontsize=\scriptsize, linenos]{c}
#include <stdio.h>

long int lof (FILE * pFile) {
  long int oldPos = ftell(pFile);
  
  fseek(pFile, 0L, SEEK_END);
  long int endPos = ftell(pFile);
  
  fseek(pFile, oldPos, SEEK_SET);
  return endPos;
}

int main (void) {
  FILE * pFile = fopen("file.txt", "r");  
  printf(
    "file.txt: %ld bytes.\n", lof(pFile)
  );
  fclose(pFile);
}
\end{minted}
\end{codebox}
\end{columns}
%
\end{frame}
