% =========================================================================== %

\begin{frame}[t,plain]
\titlepage
\end{frame}

% =========================================================================== %

\begin{frame}{Recap}
%
Letzte Stunde haben wir gesehen/kennengelernt:
%
\begin{columns}[T]
\column{.5\linewidth}
\begin{itemize}
\item Sprunganweisung \mintinline{c}{goto} -- Spaghetti-Code
\item Schleifen
	\begin{itemize}
	\item \mintinline{c}{while} und \mintinline{c}{do-while} -- allgemeine Bedingungen
	\item \mintinline{c}{for} -- Zählschleifen
	\item Eingriff mit \mintinline{c}{continue} und \mintinline{c}{break}
	\end{itemize}
\item Die CPP-Referenz \url{https://en.cppreference.com/}
\end{itemize}
%
\column{.5\linewidth}
\begin{itemize}
\item Automatische Listen -- Arrays
	\begin{itemize}
	\item Speicherbild
	\item Syntax für Zugriff -- Ziel-Operator und Index-Operator
	\item Initialisierung
	\item Mehrdimensionale Arrays
	\end{itemize}
\item (Pseudo-)Zufallszahlen
\item \emph{Fragen hierzu?}
\end{itemize}
\end{columns}
%
\end{frame}

% =========================================================================== %

\begin{frame}[fragile]{Aus den letzten Übungen}
%
\begin{codebox}[Aufgabe \enquote{Weihnachtsbaum} -- Drei Schleifen]
\begin{minted}[fontsize=\scriptsize,linenos]{c}
#include <stdio.h>
#include <stdlib.h>

int main () {
  int i, j;
  int h = 7;
  
  for (i = 0; i < h; i++) {
    for (j = 0; j < (h-i)  ; j++) {printf(" ");}
    for (j = 0; j < 2*i + 1; j++) {printf("*");}
    printf("\n");
  }
}
\end{minted}
\end{codebox}
%
\end{frame}

% =========================================================================== %

\begin{frame}[fragile]{Aus den letzten Übungen}
%
\begin{codebox}[Aufgabe \enquote{Weihnachtsbaum} -- Zwei Schleifen]
\begin{minted}[fontsize=\scriptsize,linenos]{c}
#include <stdio.h>
#include <stdlib.h>

int main () {
  int r, c, h = 7;
  
  for   (r = 0; r <     h    ; r++) {
    for (c = 0; c < 2 * h + 1; c++) {
      printf("%s", abs(c - h) < (r + 1)   ?   "*"   :   " ");
    }
    printf("\n");
  }
}
\end{minted}
\end{codebox}
%
\end{frame}

% =========================================================================== %

\begin{frame}{Script}
%
\begin{itemize}
\item Kapitel 8
	\begin{itemize}
	\item 8.3. C-Strings
		\begin{itemize}
		\item 8.3.4. Nützliche Funktionen aus der String-Library
		\end{itemize}
	\end{itemize}
\end{itemize}
%
\end{frame}

% =========================================================================== %

\begin{frame}[fragile]{C-String library -- \texttt{string.h}}
%
\tcbset{
	width=.48\linewidth,
	on line
}
%
Sammlung von Routinen zum Umgang mit Strings\newline
Für uns: \emph{Byte-Strings}. Siehe: \url{http://de.cppreference.com/w/c/string/byte}
\begin{center}		% begin..end creates a new paragraph, a behaviour that is otherwise inactive in
\hrule				% the beamer class. So I abuse this command here -- nothing is actually centered.
\end{center}
%
\begin{tcolorbox}[title=\texttt{strlen}, height=4.7cm]
\begin{itemize}
\item Gibt Zahl der Bytes bis zum ersten Auftreten eines \emph{Null-Chars} aus.
\item[$\Rightarrow$] Zahl der Buchstaben in einem String
\item[$\Rightarrow$] Keine \emph{sinnvolle} verwendung mit anderen Datentypen
\end{itemize}
\end{tcolorbox}
%
\begin{tcolorbox}[title=\texttt{sizeof}, height=4.7cm]
\begin{itemize}
\item Gibt je nach Datentyp verschiedene Werte zurück.
\item Automatische Arrays: Größe des Arrays in Bytes
	\begin{itemize}
	\item Bei Strings also: Schriftzeichen inclusive Null-Char
	\end{itemize}
\item Ansonsten: Größe des \emph{Datentyps}
\end{itemize}
\end{tcolorbox}
%
\end{frame}

% =========================================================================== %

\begin{frame}[fragile]
%
\begin{columns}[T]
\column{.6\linewidth}
\begin{codebox}
\begin{minted}[fontsize=\scriptsize,linenos]{c}
#include <stdio.h>
#include <string.h>

int main () {
   char array[] = {1, 2, 3, 0, 4};
   int  iList[] = {1, 2, 3, 0, 4};
   
   // lu: unsigned long int
   printf("sizeof(array): %lu\n", sizeof(array));
   printf("strlen(array): %lu\n", strlen(array));
   printf("sizeof(int)  : %lu\n", sizeof(int)  );

   // unsinniger Ausdruck:
   //printf("strlen(int)  : %lu\n", strlen(int));
	
   printf("strlen(iList): %lu\n", strlen(iList));
   printf("sizeof(iList): %lu\n", sizeof(iList));
}
\end{minted}
\end{codebox}
%
\column{.35\linewidth}
\begin{cmdbox}[Ausgabe]
\ttfamily \scriptsize
sizeof(array): 5\newline
strlen(array): 3\newline
sizeof(int)\quad: 4\newline
strlen(iList): 1\newline
sizeof(iList): 20\newline
\end{cmdbox}
%
\begin{warnbox}
Warnung in Zeile 16:\newline
\texttt{int *} passt nicht auf\newline
\texttt{char *}!
\end{warnbox}
%
\end{columns}
%
\end{frame}

% =========================================================================== %

\begin{frame}[fragile]{Texte auf Gleichheit Prüfen -- \texttt{strcmp}}
%
\begin{columns}[T]
\column{.4\linewidth}
\begin{codebox}[Syntax]
\footnotesize\texttt{strcmp(ptr\_A, ptr\_B)}
\end{codebox}
%
\begin{tcolorbox}[title=Rückgabewerte]
\begin{itemize}
\item 0 bei Gleichheit
\item kleiner 0, wenn A lexikographisch${}^{1}$ vor B
\item größer 0, wenn A lexikographisch nach B
\end{itemize}
\end{tcolorbox}
%
\column{.6\linewidth}
\begin{codebox}[Beispiel]
\begin{minted}[fontsize=\scriptsize,linenos]{c}
#include <stdio.h>
#include <string.h>

int main () {
  char s[] = "Hieronymus";
  
  printf("%d\n", strcmp(s, "Hieronymus")); //  0
  printf("%d\n", strcmp(s, "hieronymus")); //-32
  printf("%d\n", strcmp(s, "HieronYmus")); // 32
  printf("%d\n", strcmp(s, "Lauser")    ); // -4
}
\end{minted}
\end{codebox}
\scriptsize ${}^{1}$ \enquote{alphabetisch}, berücksichtigt Ziffern und Sonderzeichen
%
\end{columns}
%
\end{frame}

% =========================================================================== %

\begin{frame}[fragile]{Strings Verketten -- \texttt{strcat}}
%
\begin{columns}[T]
\column{.5\linewidth}
\begin{codebox}[Syntax]
\footnotesize\texttt{strcat(ptr\_dst, ptr\_src)}
\end{codebox}
%
\begin{itemize}
\item Hängt String bei \texttt{ptr\_src} an das Ende des Strings bei \texttt{ptr\_dst} an
\item Bei \texttt{ptr\_dst} muss bereits genug Speicher zur Verfügung stehen
\item String bei \texttt{ptr\_src} bleibt unverändert
\item Rückgabewert: \texttt{ptr\_dst}
\end{itemize}
%
\column{.5\linewidth}
\begin{codebox}[Beispiel]
\begin{minted}[fontsize=\scriptsize,linenos]{c}
#include <stdio.h>
#include <string.h>

int main () {
  char dst[256] = "Hieronymus";
  char src[   ] = " Lauser";

  strcat(dst, src);
  printf("%s\n", dst);
}
\end{minted}
\end{codebox}
%
\begin{cmdbox}[Ausgabe]
\scriptsize\texttt{Hieronymus Lauser}
\end{cmdbox}
%
\end{columns}
%
\end{frame}

% =========================================================================== %

\begin{frame}[fragile]{Strings Kopieren -- \texttt{strcpy}}
%
\begin{columns}[T]
\column{.5\linewidth}
\begin{codebox}[Syntax]
\footnotesize\texttt{strcpy(ptr\_dst, ptr\_src)}
\end{codebox}
%
\begin{itemize}
\item Kopiert Inhalt von String \texttt{ptr\_src} nach \texttt{ptr\_dst}
\item Für \texttt{ptr\_dst} muss bereits genug Speicher zur Verfügung stehen
\item \texttt{ptr\_src} bleibt unverändert
\item Rückgabewert: \texttt{ptr\_dst}
\end{itemize}
%
\column{.5\linewidth}
\begin{codebox}[Beispiel]
\begin{minted}[fontsize=\scriptsize,linenos]{c}
#include <stdio.h>
#include <string.h>

int main () {
  char dst[256];
  char src[   ] = "Hieronymus";

  strcpy(dst, src);
  printf("%s --> %s\n", src, dst);
}
\end{minted}
\end{codebox}
%
\begin{cmdbox}[Ausgabe]
\scriptsize\texttt{Hieronymus -{}-> Hieronymus}
\end{cmdbox}
%
\end{columns}
%
\end{frame}

% =========================================================================== %

\begin{frame}[fragile]{Strings durchsuchen -- \texttt{strchr} und \texttt{strrchr}}
%
\begin{columns}[T]
\column{.5\linewidth}
\begin{codebox}[Syntax]
\footnotesize\texttt{ strchr(ptr\_data, int\_code)}\newline
\footnotesize\texttt{strrchr(ptr\_data, int\_code)}
\end{codebox}
%
\begin{itemize}
\item Sucht das erste Zeichen mit Code \texttt{int\_code} im String \texttt{ptr\_data}
\item Rückgabe: Pointer auf dieses Zeichen oder \texttt{NULL} zurück, wenn nicht gefunden.
\item \texttt{strrchr}: Suche \enquote{von hinten}
\end{itemize}
%
\column{.5\linewidth}
\begin{codebox}[Beispiel]
\begin{minted}[fontsize=\scriptsize,linenos]{c}
#include <stdio.h>
#include <string.h>

int main () {
  char  str[] = "Replace *asteriscs*";
  char *found;   
   
  found = strchr(str, '*');
  while (found) {
    *found = '"';
    found = strchr(str, '*');
  }
  
  printf("%s\n", str);
}
\end{minted}
\end{codebox}
%
\end{columns}
%
\end{frame}

% =========================================================================== %

\begin{frame}[fragile]{Varianten für allgemeine Arrays: \texttt{memXXX}}
%
\begin{columns}[T]
\column{.45\linewidth}
\begin{itemize}
\item Strings: \emph{nullterminiert}
\item Zahlen-Arrays können 0 enthalten
\item Varianten \texttt{memcmp}, \texttt{memcpy}, \texttt{memchr}
\item Brauchen zusätzlichen Parameter \texttt{bytecount}
\end{itemize}
%
\begin{warnbox}
\begin{itemize}
\item Größe der Datentypen beachten!
\item Genug Speicher zur Verfügung stellen!
\end{itemize}
\end{warnbox}
%
\column{.55\linewidth}
\begin{codebox}[Beispiel -- \enquote{\texttt{memcat}} mit \texttt{memcpy}]
\begin{minted}[fontsize=\scriptsize,linenos]{c}
#include <stdio.h>
#include <string.h>

int main () {
  int i;
   
  int dst[256] = {1, 2, 3};
  int src[   ] = {4, 5, 6};
  int *dummy = dst + 3;
   
  memcpy(dummy, src, 3 * sizeof(int));
   
  for (i=0; i<6; i++) {
    printf("dst[%d]: %d\n", i, dst[i]);
  }
}
\end{minted}
\end{codebox}
\end{columns}
%
\end{frame}

% =========================================================================== %

\begin{frame}[fragile]{Groß- und Kleinschreibung -- \texttt{tolower} und \texttt{toupper}}
%
\begin{columns}[T]
\column{.45\linewidth}
\begin{codebox}[Syntax]
\texttt{tolower(int\_code)}\newline
\texttt{toupper(int\_code)}
\end{codebox}
\begin{itemize}
\item Geben ASCII-Codes zurück
\item Ändern nur a...z bzw. A...Z
\item Keine Behandlung von Umlauten, Akzenten, ...
\item Definiert in \texttt{ctype.h}
\end{itemize}
%
\column{.55\linewidth}
\begin{codebox}[Beispiel -- \enquote{\texttt{memcat}} mit \texttt{memcpy}]
\begin{minted}[fontsize=\scriptsize,linenos]{c}
#include <stdio.h>
#include <ctype.h>

int main () {
  char src[] = "MiXeD cAsE, äöüß & ÄÖÜ";
  char upC[256], loC[256];
  int i;

  for (i=0; src[i]; i++) {
    upC[i] = toupper(src[i]);
    loC[i] = tolower(src[i]);
  }
  printf("original  : %s\n", src);
  printf("upper case: %s\n", upC);
  printf("lower case: %s\n", loC);
}
\end{minted}
\end{codebox}
\end{columns}
%
\end{frame}

% =========================================================================== %

\begin{frame}[fragile]{Zeichentypen in Klassen einteilen -- \texttt{isXXX}-Functions}
%
\begin{columns}[T]
\column{.37\linewidth}
\begin{codebox}[Syntax]
\texttt{isXXX(int\_code)}
\end{codebox}
\begin{itemize}
\item Ganze Klasse von Funktionen
\item Rückgabe: Wahrheitswert
\item \texttt{int\_code} vom Typ XXX?
\item Ziffer, Buchstabe, ...
\item Siehe Tabelle auf\newline
	{\footnotesize\url{http://de.cppreference.com}}
\item Definiert in \texttt{ctype.h}
\end{itemize}
%
\column{.63\linewidth}
\begin{codebox}[Beispiel -- Ist Text Fließkommazahl?]
\begin{minted}[fontsize=\scriptsize,linenos]{c}
#include <stdio.h>
#include <ctype.h>
#include <string.h>

int main () {
  char input[256]; int i, tVal = 1;
  printf("Please enter a number:\n");
  scanf("%255s", input);
	
  for (i=0; i<strlen(input); i++)
    tVal &= isdigit(input[i]) || (input[i] == '.');
	
  printf("%s is%s a number.\n", 
         input, 
         tVal ? "" : " not");
}
\end{minted}
\end{codebox}
\end{columns}
%
\end{frame}

% =========================================================================== %

\begin{frame}[fragile]{Umwandlung Zahlenwert $\leftrightarrow$ String}
%
\begin{columns}[T]
\column{.5\linewidth}
\begin{codebox}[String $\rightarrow$ Zahl]
\footnotesize\texttt{number = atoXXX(ptr\_String)}
\end{codebox}
\begin{itemize}
\item ASCII to XXX
\item \texttt{atoi}: int
\item \texttt{atof}: float
\item \texttt{atoll}: long long
\item[$\Rightarrow$] \url{http://de.cppreference.com/w/c/string/byte/atoi}
\item Definiet in \texttt{stdlib.h}
\end{itemize}
%
\column{.5\linewidth}
\begin{codebox}[Zahl $\rightarrow$ String]
\begin{minted}[fontsize=\scriptsize]{c}
sprintf(ptr_String, 
        "format string", 
        number1, number2, ...
);
\end{minted}
\end{codebox}
\begin{itemize}
\item Ausgabe nicht auf Bildschirm, sondern \enquote{in String}
\item Format String wie bei \texttt{printf}
\item Auch mehrere Werte gleichzeitig
\item Definiert in \texttt{stdio.h}
\end{itemize}
\end{columns}
%
\end{frame}

% =========================================================================== %

\begin{frame}[fragile]
%
\begin{columns}[T]
\column{.5\linewidth}
\begin{codebox}[atoi{,} atof]
\begin{minted}[fontsize=\scriptsize,linenos]{c}
#include <stdio.h>
#include <stdlib.h>

int main () {
   char str[] = "1234.56789";

   double d;
   int    i;

   i = atoi(str);
   d = atof(str);

   printf("%s -> int   : %d\n" , str, i);
   printf("%s -> double: %lf\n", str, d);
}
\end{minted}
\end{codebox}
%
\column{.5\linewidth}
\begin{codebox}[sprintf]
\begin{minted}[fontsize=\scriptsize,linenos]{c}
#include <stdio.h>

int main () {
  int day, month, year;
  char filename[256];
	
  printf("Please enter today's date:\n");
  scanf("%d %d %d", &year, &month &day);
	
  sprintf(filename, 
    "logfile_%04d-%02d-%02d.log",
    year, month, day);
    
  printf("Writing into log file:\n%s", 
     filename);
}
\end{minted}
\end{codebox}
\end{columns}
%
\end{frame}
