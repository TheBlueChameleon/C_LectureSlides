% =========================================================================== %

\begin{frame}[t,plain]
\titlepage
\end{frame}

% =========================================================================== %

\begin{frame}{Recap: Speicherstrukturen}
%
\begin{columns}[T]
\column{.5\linewidth}
\begin{itemize}
\item \mintinline{c}{struct}s
	\begin{itemize}
	\item \enquote{Container-Datentyp}
	\item Daten-Attribute
	\item Pointer-Attribute
	\item Pointer auf \inC{struct}s
	\item Operator \texttt{->}
	\end{itemize}
\item \mintinline{c}{enum}s
	\begin{itemize}
	\item automatisch nummerierte \mintinline{c}{int}-Liste
	\item Konstanten, die Programm gut lesbar/wartbar machen
	\end{itemize}
\end{itemize}
%
\column{.55\linewidth}
\begin{itemize}
\item \mintinline{c}{union}s
	\begin{itemize}
	\item Casting-Interface
	\item Syntax wie \mintinline{c}{struct}
	\end{itemize}
\item \mintinline{c}{typedef}
	\begin{itemize}
	\item Alias für häufig benutzte Datentypen
	\item Mehrfach-Definitionen, \zB für Pointer
	\end{itemize}
\end{itemize}
\end{columns}
%
\end{frame}

% =========================================================================== %

\begin{frame}{Recap: Modularisierung}
%
\begin{columns}[T]
\column{.5\linewidth}
\begin{itemize}
\item Aufteilung in Module
	\begin{itemize}
	\item Zweck: Gliederung, Kompilierzeit sparen, Closed-Source-Lösungen
	\item Header: Definitionen; Sichtbar-Machen
	\item Module: Normaler C-Code
	\item Globale Variablen über Module hinweg: \mintinline{c}{extern}
	\item Ggf. Kompilieren und Linken über spezielle Aufrufe
	\end{itemize}
\end{itemize}
%
\column{.5\linewidth}
\begin{itemize}
\item Modifier, Speicherklassen
	\begin{itemize}
	\item \mintinline{c}{static}, \mintinline{c}{extern}, \mintinline{c}{register}
	\item \mintinline{c}{unsigned}, \mintinline{c}{const}, \mintinline{c}{restrict}, 
		\mintinline{c}{volatile}
	\end{itemize}
\end{itemize}
\end{columns}
%
\begin{center}
\emph{Noch Fragen?}
\end{center}
%
\end{frame}

% =========================================================================== %

\begin{frame}{Script}
%
\begin{itemize}
\item Kapitel 9
	\begin{itemize}
	\item 9.4. Funktionszeiger
	\end{itemize}
\end{itemize}
%
\end{frame}

% =========================================================================== %

\begin{frame}[fragile]{Wiederholung: Pointer vs. Skalar}
%
\begin{columns}[T]
\column{.5\linewidth}
%
\begin{itemize}
\item Skalar: Tatsächliches Objekt -- einzelne Zahl
\item Pointer: Wo ist die gefragte Zahl?
\item Analogie: Früchte in nummerierten Boxen
	\begin{itemize}
	\item Frucht         \tab \thus Skalar
	\item Nummer der Box \tab \thus Pointer 
	\item Nummer der Box bestimmen \newline
		\Thus Adress-Operator \texttt{\&variable}
	\item Inhalt der Box ansehen oder verändern \newline
		\Thus Dereferenzierung \texttt{*variable}
	\end{itemize}
\end{itemize}
%
\column{.5\linewidth}
\vspace{-10pt}
\begin{codebox}[Beispiel: Einfache Pointer]
\begin{minted}[fontsize=\scriptsize, linenos]{c}
#include <stdio.h>

int main () {
  int    x =  7;
  int * px = &x;
  
  printf("%d\n",   x);   // 7
  printf("%d\n", *px);   // 7
  
  *px = 8;
  printf("%d\n",   x);   // 8
  
  x = 9;
  printf("%d\n", *px);   // 9
}
\end{minted}
\end{codebox}
\end{columns}
%
\end{frame}

% =========================================================================== %

\begin{frame}[fragile]{Wiederholung: Automatische Arrays}
%
\begin{columns}[T]
\column{.4\linewidth}
\begin{itemize}
\item Liste von Werten
\item Realisierung über Startadresse und Index
\item Analogie mit Früchten und Boxen:
	\begin{itemize}
	\item Nummer der ersten Box \newline 
		$\rightarrow$ Pointer\\
		\vspace{3pt}
	\item Schritte zwischen erster und gewünschter Box \newline
		$\rightarrow$ Index\\
		\vspace{3pt}
	\item Eine bestimmte Frucht ansehen oder austauschen \newline
		$\rightarrow$ \texttt{variable[index]}
	\end{itemize}
\end{itemize}
%
\column{.6\linewidth}
\vspace{-13pt}
\begin{codebox}[Beispiel: Statische Arrays]
\begin{minted}[fontsize=\scriptsize, linenos]{c}
#include <stdio.h>

int main () {
  int px[] = {1, 2, 3};

  printf("first item at %p\n", px);
  printf("value first item: %d\n", px[0]);
  printf("alternative syntax: %d\n", *px);
}
\end{minted}
\end{codebox}
%
\begin{hintbox}
\small
Schon bei Deklaration genau überlegen: ist das ein Pointer oder ein Skalar?\newline
\Thus Pointer im Namen kennzeichnen!
\end{hintbox}
\end{columns}

%
\end{frame}

% =========================================================================== %

\begin{frame}[fragile]{Wiederholung: Dynamische Arrays}
%
\begin{columns}[T]
\column{.5\linewidth}
\begin{itemize}
\item Box-Anzahl erst zur Laufzeit bestimmen
\item Analogie mit Früchten und Boxen:
	\begin{itemize}
	\item Gib mir $N$ Boxen \newline 
		\Thus \footnotesize \texttt{ptr = malloc(N * sizeof(type))}\\
		\vspace{6pt}
	\item \small Gib mir $N$ \emph{leere} Boxen \newline 
		\Thus \footnotesize \texttt{ptr = calloc(N, sizeof(type))}\\
		\vspace{6pt}
	\item \small Ich brauche meine Boxen nicht mehr \newline
		\Thus \footnotesize \texttt{free(ptr)}\\
		\vspace{6pt}
	\item \small Zahl nötiger Boxen hat sich geändert \newline
		\Thus \footnotesize \texttt{ptr = realloc(ptr, N * size)}
	\end{itemize}
\end{itemize}
%
\column{.5\linewidth}
\vspace{-5pt}
\begin{codebox}[Beispiel: Dynamische Arrays]
\begin{minted}[fontsize=\scriptsize, linenos]{c}
#include <stdlib.h>

int main () {
  int * px = malloc(5 * sizeof(*px));

  for (int i=0; i<5; ++i) {px[i] = i;}

  px = realloc(px, 6 * sizeof(*px));

  free(px);
}
\end{minted}
\end{codebox}
\end{columns}
%
\end{frame}

% =========================================================================== %

\begin{frame}[fragile]{Wiederholung: Funktionen -- Argumente}
%
\begin{columns}[T]
\column{.5\linewidth}
\begin{itemize}
\item Jede Funktion hat \enquote{ihre eigenen} Boxen
\item Aufruf mit Parametern: Vorbefüllen der Boxen der Funktion
\item Ausnahme: Globale Variablen
\item Werte durch aufrufend einer Funktion ändern \newline
	\Thus Pointer übergeben: welchen Wert ändern
\end{itemize}
%
\begin{warnbox}
\small
Das Ziel des Pointers muss zu beim Zugriff existieren!
\end{warnbox}
%
\column{.5\linewidth}
\vspace{-10pt}
\begin{codebox}[Beispiel: Parameterübergabe]
\begin{minted}[fontsize=\scriptsize, linenos]{c}
void passPtr(int x, int * px) {
   printf("&x: %p,  px: %p\n", &x,  px);
   printf(" x: %d, *px: %d\n",  x, *px);
   
     x = 7;
   *px = 7;
}

int main (void) {
   int    y = 5, z = 6;
   int * pz = &z;
   
   printf("&y: %p,  pz: %p\n", &y,  pz);
   passPtr(y, pz);
   printf(" y: %d, *pz: %d\n",  y, *pz);
}
\end{minted}
\end{codebox}
\end{columns}

%
\end{frame}

% =========================================================================== %

\begin{frame}[fragile]{Function Pointers: Motivation}
%
\begin{columns}[T]
\column{.5\linewidth}
\begin{itemize}
\item Idee: Eine Funktion soll parametrisiert verschiedene andere Funktionen aufrufen
\item Beispiel 1: Buttons
	\begin{itemize}
	\item User-Interface mit mehreren Buttons
	\item Anwender soll über diese seine wichtigsten Programm-Funktionen aufrufen können
	\item Funktion des Buttons also zur Laufzeit änderbar
	\end{itemize}
\item Beispiel 2: aus der Mathe:
	\begin{itemize}
	\item Eine erste Funktion berechnet tatsächlich Funktionswert $f(x)$
	\item Eine zweite Funktion soll Nullstellen von $f(x)$ finden
	\end{itemize}

\end{itemize}
%
\column{.5\linewidth}
\begin{itemize}
\item Lösung: Variable enthält Information: Welche Funktion aufrufen
\item Problem: Information \enquote{Funktionsname} geht beim Kompilieren verloren
\item Lösung: Pointer auf die Speicherstelle, wo die Funktion beginnt
\item[$\Rightarrow$] \enquote{Springe an die Speicherstelle [Pointer], lese die Bytes dort und führe die Anweisungen dort aus.}
\end{itemize}
\end{columns}
%
\end{frame}

% =========================================================================== %

\begin{frame}[fragile]{Funktionszeiger: Umsetzung}
%
\begin{columns}[T]
\column{.5\linewidth}
\begin{itemize}
\item Compiler muss über zum Funktionszeiger dieselben Informationen \enquote{wissen} wie über Funktion selbst:
	\begin{itemize}
	\item Rückgabetyp
	\item Name und Art der Parameter
	\item Name des Pointers
	\end{itemize}
\item Vergleiche \emph{Funktions-Prototypen} (Deklaration von Funktionen, deren Körper noch nicht geschrieben wurde, um sie bereits aufrufbar zu machen)
\end{itemize}
%
\column{.5\linewidth}
\begin{codebox}[Syntax: Deklaration]
\begin{minted}[fontsize=\scriptsize]{c}
Rueckgabetyp (*pointer) (Parameterliste);
\end{minted}
\end{codebox}
%
\begin{codebox}[Syntax: Aufruf]
\begin{minted}[fontsize=\scriptsize]{c}
pointer(Parameterliste);
\end{minted}
\end{codebox}
%
\begin{codebox}[Syntax: Zuweisung]
\begin{minted}[fontsize=\scriptsize]{c}
pointer = andere_funktion;
\end{minted}
\end{codebox}
\end{columns}

%
\end{frame}

% =========================================================================== %

\begin{frame}[fragile]
%
\begin{codebox}[Beispiel: Funktionszeiger]
\begin{minted}[fontsize=\scriptsize, linenos]{c}
#include <stdio.h>

void foo() {
  printf("blerb.\n");
}

int main () {
  void (*funcPtr)() = foo;
  
  printf("direkt : ");
  foo();
  
  printf("funcPtr: ");
  funcPtr();
}
\end{minted}
\end{codebox}
%
\end{frame}

% =========================================================================== %

\begin{frame}[fragile]
%
\tcbset{width=.495\linewidth, on line, height=8cm}
\begin{codebox}[Beispiel: Auf Nullstellen prüfen]
\begin{minted}[fontsize=\scriptsize, linenos]{c}
#include <stdio.h>
#include <math.h>

int hasZeros(
  double(*f)(double), 
  double a, double b, double s 
) {
  if (f(a) == 0) {return 1;}
  
  int lastsgn = (f(a) > 0);
  
  for (double x=a; x<b; x+=s) {
    if (lastsgn != (f(x) > 0)) {
      return 1;
    } else {
      lastsgn = (f(x) > 0);
    }
  }
  return 0;
}
\end{minted}
\end{codebox}
%
\begin{codebox}[...Fortsetzung]
\begin{minted}[fontsize=\scriptsize, linenos, firstnumber=last]{c}
int main (void) {
  printf("sin has %szeros %s\n",
    hasZeros(sin, 0, 3.14, 0.01) ? 
    "" : "no ",
    "between 0 and pi."
  );
  
  printf("cos has %szeros %s\n",
    hasZeros(cos, 0, 3.14, 0.01) ? 
    "" : "no ",
    "between 0 and pi."
  );
  
  printf("exp has %szeros %s\n",
    hasZeros(exp, 0, 3.14, 0.01) ? 
    "" : "no ",
    "between 0 and pi."
  );
}
\end{minted}
\end{codebox}
%
\end{frame}

% =========================================================================== %

\begin{frame}[fragile]{\texttt{qsort}}
%
\begin{columns}[T]
\column{.5\linewidth}
\begin{itemize}
\item Vorgefertigte Sortierfunktion
\item Quicksort-Algorithmus: Besprochen in VL \emph{Algorithmen und Datenstrukturen}
\item Sortieren von allgemeinen Listen
	\begin{itemize}
	\item unabhängig vom Datentyp
	\item für beliebige Sortier-Kriterien
	\end{itemize}
\end{itemize}
%
\column{.5\linewidth}
\begin{itemize}
\item Lösung über Funktionszeiger für Sortier-Kriterium
\item \mintinline{c}{void *} auf zu sortierende Daten -- allgemeingültig
\item Zusätzliche Information: Größe eines Elements
\item \mintinline{c}{size_t}: Alias für \mintinline{c}{unsigned long}  
\end{itemize}
\end{columns}
%
\begin{codebox}[Prototyp der Funktion \texttt{qsort}]
\begin{minted}[fontsize=\scriptsize]{c}
void qsort( void * ptr, size_t count, size_t size,
            int (*comp)(const void *, const void *) );
\end{minted}
\end{codebox}
%
\end{frame}

% =========================================================================== %

\begin{frame}[fragile]
%
\tcbset{width=.495\linewidth, on line}
%
\begin{codebox}[Beispiel: \texttt{qsort}, equal height group=Gqsort]
\begin{minted}[linenos, fontsize=\scriptsize]{c}
#include <stdio.h>
#include <stdlib.h>
 
int asc(const void* a, 
        const void* b
       ) {
  int arg1 = *(const int *) a;
  int arg2 = *(const int *) b;
 
  if (arg1 < arg2) return -1;
  if (arg1 > arg2) return  1;
  return 0;
}

int dsc(const void* a, 
        const void* b
       ) {
  return -comp_ascending(a, b);
}
\end{minted}
\end{codebox}
%
\begin{codebox}[...Fortsetzung, equal height group=Gqsort]
\begin{minted}[linenos, firstnumber=last, fontsize=\scriptsize]{c}
int main(void) {
  int ints[] = {-2, 99, 0, -743, 2, 4};
  int size = sizeof(ints)/sizeof(*ints);
    
  printf("Aufsteigend:\n");
  qsort(ints, size, sizeof(int), asc);
    
  for (int i = 0; i < size; ++i) {
    printf("%d ", ints[i]);
  }
  printf("\n");
    
  printf("Absteigend:\n");
  qsort(ints, size, sizeof(int), dsc);
    
  for (int i = 0; i < size; ++i) {
    printf("%d ", ints[i]);
  }
  printf("\n");
}
\end{minted}
\end{codebox}
%
\end{frame}

% =========================================================================== %

\begin{frame}{Script}
%
\begin{itemize}
\item Kapitel 12
	\begin{itemize}
	\item 12.2. Kommandozeilenparameter
	\end{itemize}
\end{itemize}
%
\end{frame}

% =========================================================================== %

\begin{frame}{Kommandozeilenparameter}
%
\begin{columns}[T]
\column{.5\linewidth}
\begin{itemize}
\item Systemebene: Kommandozeile $\rightarrow$ Betriebssystem (OS): Programm starten
\item GUIs: Nur Verpackung, Weiterleitung an Konsole
\item Hierbei: Parameter $\rightarrow$ Programm
\end{itemize}
%
\begin{cmdbox}[Beispiel: Compileraufruf]
\footnotesize 
gcc myCode.c -std=c11 -Wall -lm
\end{cmdbox}
{\footnotesize \texttt{gcc} ist das Programm. Es wird mit den Parametern \texttt{myCode.c}, \texttt{-std=c11}, \texttt{-Wall} und \texttt{-lm} aufgerufen.}
%
%
\column{.5\linewidth}
\begin{itemize}
\item[$\Rightarrow$] Beliebige Zeichenketten können an ein Programm weiter gereicht werden!
\item Umsetzung: OS...
	\begin{itemize}
	\item aufrufendes Kommando $\rightarrow$ Speicher
	\item \enquote{Parsing} des Kommandos: Zerlegen in Einzelteile (Trennzeichen: Leerzeichen)
	\item Sonderbehandlung: Text in ''double quotes'' wird nicht zertrennt
	\item Übergibt Pointer auf die einzelnen Kommando-Teile an das aufgerufene Programm
	\item Gleiches Verhalten bei Windows, Linux, Mac.
	\end{itemize}
\end{itemize}
\end{columns}

%
\end{frame}

% =========================================================================== %

\begin{frame}[fragile]{Kommandozeilenparameter -- Entgegennahme}
%
\begin{columns}[T]
\column{.5\linewidth}
\begin{itemize}
\item Bisher: \mintinline{c}{int main()} -- Keine Parameter für das Hauptprogramm
\item Alternative Form: 
	\begin{codebox}[main mit Parametern]
	\footnotesize\mintinline{c}{int main(int argc, char **argv)}
	\end{codebox}
\end{itemize}
%
\column{.5\linewidth}
\begin{itemize}
\item neue Parameter:
	\begin{itemize}
	\item \mintinline{c}{int argc} -- \enquote{argument count}: Wie viele Parameter
	\item \mintinline{c}{char ** argv} -- \enquote{argument values}: Liste der Parameter
	\end{itemize}
\item Erinnerung: String ist Liste von \mintinline{c}{char}s
\item Parameterliste ist Liste von Strings
\item[$\Rightarrow$] \texttt{argv} ist \emph{Liste von Listen}
\item[$\Rightarrow$] \mintinline{c}{char **}
\end{itemize}
\end{columns}
%
\end{frame}

% =========================================================================== %

\begin{frame}[fragile]
%
\begin{codebox}[Beispiel: Alle Parameter einzeln ausgeben]\label{code:cmdLine}
\begin{minted}[fontsize=\scriptsize, linenos]{c}
#include <stdio.h>

int main (int argc, char ** argv) {
  printf("This programme goes by the file name:\n");
  printf("%s\n\n", argv[0]);
  
  if (argc == 1) {
    printf("no Parameters.\n");
  } else {
  printf("%3d Parameters:\n", argc - 1);
  }
  int i;
  for(i=1; i<argc; ++i) {
    printf("%3d\t%s\n", i, argv[i]);
  }
}
\end{minted}
\end{codebox}
%
\end{frame}

% =========================================================================== %

\begin{frame}[fragile]{Kommandozeilenparameter: Kompilieren und Aufrufen}
%
\begin{itemize}
\item Kein unterschied beim Kompilieren.
	\begin{cmdbox}[Kompilieren mit Parametern -- wie immer]
	\footnotesize 
	gcc myCode.c -std=c11 -Wall -Wextra -Wpedantic -lm -o myExecutable
	\end{cmdbox}
\item Starten durch anhängen der Parameter
	\begin{cmdbox}[Ausführen mit Parametern]
	\footnotesize 
	./myExecutable {\color{cyan} my parameters}
	\end{cmdbox}
\item Funktioniert so auf jedem Linux/Mac-Rechner
\item Windows: Programm-Aufruf bis auf führendes \texttt{./} identisch
\end{itemize}
%
\end{frame}

% =========================================================================== %

\begin{frame}[fragile]{Kommandozeilenparameter: Beispiel von Folie \ref{code:cmdLine}}
%
\begin{cmdbox}[Kompilieren ohne Script{,} Aufruf mit Parametern]
\begin{minted}[fontsize=\footnotesize]{text}
gcc cmdLineProc.c -std=c11 -Wall -Wextra -Wpedantic  -o cmdLineProc
./cmdLineProc Hieronymus Lauser "Chameleons are beautiful" more Parameters
\end{minted}
\end{cmdbox}
%
\begin{cmdbox}[Ausgabe]
\begin{minted}[fontsize=\scriptsize]{text}
This programme goes by the file name:
./cmdLineProc

  5 Parameters:
  1	Hieronymus 
  2	Lauser 
  3	Chameleons are beautiful
  4	more 
  5	Parameters
\end{minted}
\end{cmdbox}
%
\end{frame}