% =========================================================================== %

\begin{frame}[t,plain]
\titlepage
\end{frame}

% =========================================================================== %

\begin{frame}{Script}
%
\begin{itemize}
\item Kapitel 3
	\begin{itemize}
	\item 3.1 Daten im Speicher
	\item 3.1.2 Adressierung – Pointer
	\item 3.5 Zahlenformate
	\end{itemize}
\item Kapitel 4
	\begin{itemize}
	\item 4.1 Dateneingabe mit \mintinline{c}{scanf}
	\end{itemize}
\end{itemize}
%
\end{frame}

% =========================================================================== %

\begin{frame}[fragile]{Binärdarstellung}
%
\begin{columns}[T]
\column{.5\linewidth}
\begin{tcolorbox}[title=Dezimalzahlen]
\begin{itemize}
\item Nummeriere Zahlen von rechts nach links durch
\item Beginne bei null
\item Zehnerpotenz multiplizert mit Ziffer
\end{itemize}
\begin{codebox}[]
{\scriptsize \ttfamily{2\,1\,0}\newline}
$\texttt{135}_{10} = 1 \cdot 10^2 + 3 \cdot 10^1 + 5 \cdot 10^0$
\end{codebox}
\end{tcolorbox}
%
\column{.5\linewidth}
\begin{tcolorbox}[title=Binärzahlen]
\begin{itemize}
\item Selbes Prinzip
\item Nur zwei Ziffern: 1 und 0
\item \emph{Zweier}potenzen
\end{itemize}
\begin{codebox}[]
{\scriptsize \ttfamily{7\,6\,5\,4\,3\,2\,1\,0}}\newline
$\texttt{10000111}_{2}$\newline
$\quad   =   1 \cdot 2^7   +   1 \cdot 2^2   +   1 \cdot 2^1   +   1 \cdot 2^0$\newline
$\quad   =   135_{10}$
\end{codebox}
\end{tcolorbox}
\end{columns}
%
\end{frame}

% =========================================================================== %

\begin{frame}{Hexadezimal- und Oktalzahlen}
%
\begin{columns}
\column{.45\linewidth}
\begin{itemize}
\item Hexadezimalzahlen
	\begin{itemize}
	\item 16 verschiedene \enquote{Ziffern}:\\ 
		0-9, A-F
	\item Elegante Darstellung von Binärzahlen
	\item Häufig in technischen Dokumenten
	\item Immer noch: Normale Zahl
	\item Markierung: \texttt{0x} oder Suffix \texttt{h}
	\end{itemize} 
\item Oktalzahlen: 
	\begin{itemize}
	\item 8 Ziffern (0-7)
	\item Bedeutung bei Festplattentreibern (8 bits pro Byte)
	\end{itemize}
\end{itemize}
%
\column{.55\linewidth}
\begin{tcolorbox}[title=Beispiel]
\footnotesize
\begin{tabular}{cc}
\textbf{Basis}    &   \textbf{Darstellung} \tabcrlf
10 (Dezimal)      &   10\,794 \\
2 (Binär)         &   \texttt{0010101010 00101010}$_2$ \\
16 (Hexadezimal)  &   \texttt{0x2A2A} \\
8 (Oktal)         &   \texttt{25052}$_8$ \\
\end{tabular}
\end{tcolorbox}
%
\begin{hintbox}
\small
Why do programmers confuse Helloween and Christmas?\newline
Because \texttt{dec(25) == oct(31)} 
\end{hintbox}
\end{columns}
%
\end{frame}

% =========================================================================== %

\begin{frame}[fragile]{Pointer}
%
\small
%
\begin{codebox}[Syntax: Pointer-Variable Deklarieren]
	\mintinline{c}{Datentyp * Variable;}
\end{codebox}
%
\tcbset{
	colback=black!5!white,
	colframe=blue!40!black,
	leftupper=7mm,
	arc=0pt,
	width=.48\linewidth,
	on line}
\begin{tcolorbox}[title=Adress-Operator]
	\mintinline{c}{&variable}
\end{tcolorbox}
\hspace{\fill}
\begin{tcolorbox}[title=Dereferenzierungs-Operator]
	\mintinline{c}{*variable}
\end{tcolorbox}
%
\begin{itemize}
\item Pointer: Formal ein \mintinline{c}{unsigned long long int}, also eine 64bit-Zahl.
\item Vorstellung: Speicher als eindimensionale Liste; Pointer ist dann Listen-Index\newline
	(\enquote{Nummer des Bytes im Speicher})
\item Immer noch zugeordnet: Datentyp -- nötig für lesenden/schreibender Zugriff.\newline
	(\emph{Wie viele Bytes lesen/schreiben?})
\end{itemize}
%
\end{frame}

% =========================================================================== %

\begin{frame}[fragile]
%
\begin{codebox}
\begin{minted}[fontsize=\scriptsize,linenos]{c}
#include <stdio.h>

int main () {
  int   Zahl   = 7;
  int * Zeiger = &Zahl;
  
  printf("Zahl ist %d\n", Zahl);                   // 7
  printf("Zeiger zeigt auf %p\n",        Zeiger);  // 0x7ffdf335b63c
  printf("an dieser Stelle steht %d\n", *Zeiger);  // 7
  
  *Zeiger = 3;
  
  printf("Jetzt ist Zahl = %d\n", Zahl);           // 3
}
\end{minted}
\end{codebox}
%
\begin{hintbox}
\small Zeile 8 -- Ausgabe von Pointern über \texttt{\%p}
\end{hintbox}
%
\end{frame}

% =========================================================================== %

\begin{frame}{Usereingaben}
%
\begin{codebox}[Syntax]
\small
\mintinline{c}{scanf(FormatString, Pointer1, Pointer2, ...);}
\end{codebox}
\begin{itemize}
\item Eingaben von Tastatur $\rightarrow$ angegebene Adresse(n)
\item Vorkompilierte Routine in Bibliothek stdio.h -- 
	keine direkte Verbindung zu Variablen im eigenen Code
\item Daher Kommunikation mittels Pointer!
\item \texttt{FormatString} wie bei \texttt{printf}
	\begin{itemize}
	\item Strenge Unterscheidung zwischen \texttt{\%f} und \texttt{\%lf}!
	\end{itemize}
\item \texttt{ListePointer} durch Kommata getrennt.
\item Empfehlung: Nur eine Eingabe pro scanf-Zeile
\end{itemize}
%
\end{frame}

% =========================================================================== %

\begin{frame}[fragile]%{scanf -- Beispiel}
%
\begin{codebox}
\begin{minted}[fontsize=\scriptsize,linenos]{c}
#include <stdio.h>

int main () {
  int foobar = 0;
  
  printf("Bitte einen Wert eingeben\n");
  scanf ("%d", &foobar);
  
  printf("Eingegeben wurde %d\n", foobar);
}
\end{minted}
\end{codebox}
%
\begin{warnbox}
\small Ohne das den Operator \mintinline{c}{&} stürzt das Programm ab, da -- verbotenerweise -- versucht wird, an die Adresse 0 (Wert von \mintinline{c}{foobar}) zu schreiben.
\end{warnbox}
%
\end{frame}

% =========================================================================== %

\begin{frame}[fragile]{Clusterf*ck \texttt{scanf}}
%
\begin{itemize}
\item \texttt{scanf}: Anfrage an das Betriebssystem: Lese String aus dem Tastaturpuffer und interpretiere diesen
\item Wenn Text im Puffer nicht auf erwartetes Format passt: lasse Text im Puffer
\item Zeilenumbruch ist auch ein Zeichen!
\end{itemize}

\begin{tcbraster}[raster columns=2,
                  raster equal height,
                  nobeforeafter,
                  raster column skip=0.2cm]
\begin{codebox}[Szenario]
\begin{minted}[fontsize=\scriptsize,linenos]{c}
#include <stdio.h>

int main () {
  int a, b;
  char c;

  scanf("%d", &a);
  scanf("%c", &c);
  scanf("%d", &b);
  printf("%d %c %d\n", a, c, b);
}

\end{minted}
\end{codebox}
%
\begin{cmdbox}[Ausführungsbeispiel]
\begin{minted}[fontsize=\scriptsize]{text}
1
+
1 
 32767
\end{minted}

\footnotesize
\Thus \texttt{c} liest den verbleibenden Zeilenumbruch von der Eingabe von \texttt{a}

\Thus \texttt{b} \textsf{kann nicht eingegeben werden und behält den Startwert \texttt{32767} (Zufälliger Wert).}
\end{cmdbox}
\end{tcbraster}
%
\end{frame}

% =========================================================================== %

\begin{frame}[fragile]{Clusterf*ck \texttt{scanf}}
%
\begin{itemize}
\item Lösung: Ein einziges Leerzeichen
\end{itemize}

\begin{tcbraster}[raster columns=2,
                  raster equal height,
                  nobeforeafter,
                  raster column skip=0.2cm]
\begin{codebox}[Szenario]
\begin{minted}[fontsize=\scriptsize,linenos]{c}
#include <stdio.h>

int main () {
  int a, b;
  char c;

  scanf("%d", &a);
  scanf(" %c", &c);
  scanf("%d", &b);
  printf("%d %c %d\n", a, c, b);
}

\end{minted}
\end{codebox}
%
\begin{cmdbox}[Ausführungsbeispiel]
\begin{minted}[fontsize=\scriptsize]{text}
1
+
1 
 32767
\end{minted}

\footnotesize
\Thus \textsf{Vor Eingabe von} \texttt{c} \textsf{wird Tastaturpuffer geleert}

\Thus \textsf{Zahlen-Eingabe ignoriert vorangehende Sonderzeichen}

\Thus \textsf{Im Zweifel: Immer Leerzeichen voran stellen}
\end{cmdbox}
\end{tcbraster}
%
\end{frame}

% =========================================================================== %

\begin{frame}{Script}
%
\begin{itemize}
\item Kapitel 5
	\begin{itemize}
	\item 5.1 Wahrheitswerte
	\item 5.2 Bedingte Ausführung von Code: \mintinline{c}{if}
	\item 5.3 Logische Operatoren
	\item 5.4 Abprüfung vieler Fälle: \mintinline{c}{switch}
	\item 5.5 Kombinierte Fallunterscheidung und Wertzuweisung
	\end{itemize}
\item Kapitel 3
	\begin{itemize}
	\item 3.2 Bitweise Logik und Bit-Shifting
	\item 3.4 Hierarchie der Operatoren
	\end{itemize}
\end{itemize}
%
\end{frame}

% =========================================================================== %

\begin{frame}[fragile]
%
\begin{tcolorbox}[title=Vergleichsoperatoren]
\begin{table}
\newcolumntype{C}{>{\ttfamily\centering\arraybackslash} p{.1\linewidth}}
\begin{tabularx}
	{\linewidth}
	{Cc|Cc|Cc}
	
	\normalfont Code & Wirkung & 
	\normalfont Code & Wirkung & 
	\normalfont Code & Wirkung \tabcrlf
	
	== & Gleichheit   & < & kleiner als & <= & kleiner oder gleich\\
	!= & Ungleichheit & > & größer als  & >= & größer oder gleich
\end{tabularx}
\end{table}
\end{tcolorbox}
%
\begin{warnbox}
\small \emph{Doppeltes} Gleichheitszeichen! Einfaches Gleichheitszeichen ist \emph{Wertzuweisung}!
\end{warnbox}
%
\begin{hintbox}
\small Reihenfolge in \mintinline{c}{>=, <=} wie beim Sprechen: Größer/Gleich
\end{hintbox}
%
\end{frame}

% =========================================================================== %

\begin{frame}[fragile]{Wahrheitswerte}
%
\begin{itemize}
\item Konzept: \emph{ja oder nein}
\item Jeder beliebige Zahlenwert
	\begin{itemize}
	\item Entweder gleich 0 -- falsch
	\item Oder ungleich 0 -- wahr
	\end{itemize}
\item Ergebnis der Vergleichsoperatoren ist ein Wahrheitswert (0 oder 1)
\item \emph{Kein} eigener Datentyp, sondern ein \mintinline{c}{int}
\item \enquote{Rechnungen} damit möglich (wenn auch selten sinnvoll)
\end{itemize}
%
\begin{codebox}[Beispiel]
\begin{minted}[fontsize=\scriptsize]{c}
int x = 17;
int truth = ((x > 5) + (x < 5) + (x == 17)) * 21;
\end{minted}
\end{codebox}
%
\end{frame}

% =========================================================================== %

\begin{frame}[fragile]{Bedingungen -- \mintinline{c}{if}, \mintinline{c}{else}}
%
\begin{columns}[T]
\column{.5\linewidth}
\begin{itemize}
\item \emph{Wenn \texttt{Bedingung} erfüllt ist, dann mache \texttt{Anweisungen}}.
\item Vermutlich DAS WICHTIGSTE Konzept. Existiert so in \emph{jeder} Sprache
\item \emph{Bedingung}: Ausdruck, der zu Wahrheitswert ausgewertet werden kann
\item \mintinline{c}{else}-Block optional: Umsetzung von \emph{Sonst mache \texttt{andere Anweisungen}}
\end{itemize}
%
\column{.45\linewidth}
\begin{codebox}[Syntax]
\begin{minted}[fontsize=\scriptsize]{c}
if (Bedingung) {
  Anweisung_1;
  Anweisung_2;
  ...
} else {
  Anweisung_3;
  ...
}
\end{minted}
\end{codebox}
\end{columns}
%
\begin{warnbox}
\small (Runde Klammern um die Bedingung), \{geschweiften Klammern um Anweisungen\}
\end{warnbox}
%
\end{frame}

% =========================================================================== %

\begin{frame}[fragile]
%
\begin{codebox}[\mintinline{text}{if} -- Beispiel]
\begin{minted}[fontsize=\scriptsize,linenos]{c}
#include <stdio.h>

int main () {
  int foobar = 0;
  
  printf("Bitte einen Wert eingeben\n");
  scanf ("%d", &foobar);
  
  if (foobar == 0) {
    printf("Betrag kleiner als Epsilon\n");
  } else {
    printf("Ave Imperator, morituri te salutant\n");
  }
}
\end{minted}
\end{codebox}
%
\begin{hintbox}
\small Code durch Einrücken übersichtlich halten!
\end{hintbox}
%
\end{frame}

% =========================================================================== %

\begin{frame}[fragile]
%
\begin{codebox}[Vergleiche -- Ein häufiger Fehler]
\begin{minted}[fontsize=\scriptsize,linenos]{c}
#include <stdio.h>

int main () {
  int foobar = 0;
  
  printf("Bitte einen Wert eingeben\n");
  scanf ("%d", &foobar);
  
  if (foobar = 0) {
    printf("Betrag kleiner als Epsilon\n");
  } else {
    printf("Ave Imperator, morituri te salutant\n");
  }
}
\end{minted}
\end{codebox}
%
\begin{warnbox}
\small Compiler-Warnung zu Zeile 9
\end{warnbox}
%
\end{frame}

% =========================================================================== %

\begin{frame}[fragile]
%
\begin{cmdbox}[Compiler-Warnung bei obigem Code]
\begin{minted}[fontsize=\scriptsize]{text}
myProgram.c: In function ‘main’:
myProgram.c:9:7: warning: suggest parentheses around assignment used as truth value 
[-Wparentheses]
   if (foobar = 0) {
       ^~~~~~
\end{minted}
\end{cmdbox}
%
\begin{hintbox}[Weiterführende Infos / Ausblicke]
\begin{itemize}
\item Selten: Wertzuweisung und gleichzeitige Fallprüfung gewünscht.
\item Beispiel: Funktion berechnet Wert, oder gibt bei Fehler 0 zurück
\item Bevorzugt: Zwei Zeilen
\item Auch erlaubt: \newline
	\mintinline{c}{if ((variable = funktion())) ... }
\end{itemize}
\end{hintbox}
%
\end{frame}

% =========================================================================== %

\begin{frame}{Logische Verknüpfung}
%
\begin{columns}[T]
\column{.45\linewidth}
\begin{itemize}
\item Kombination mehrerer Bedingungen
\item \emph{Wenn [Opfer in Sicht] UND [keine weiteren Zeugen] dann [Messer]}
\item weitere Verknüpfungen: \emph{oder} und \emph{Negation (\enquote{nicht})}
\end{itemize}
%
\column{.45\linewidth}
\begin{tcolorbox}[title=logische Vernküpfungen]
\begin{center}
\begin{table}
	\newcolumntype{C}{>{\ttfamily\centering\arraybackslash} p{.4\linewidth}}
	\newcolumntype{P}{>{         \centering\arraybackslash} p{.4\linewidth}}
\begin{tabularx}
	{.9\linewidth}	
	{PC}
Verknüpfung & \normalfont Code \tabcrlf

AND      & \&\&  \\
OR       & ||    \\
Negation & !
\end{tabularx}
\end{table}
\end{center}
\end{tcolorbox}
\end{columns}
%
%\tcbset{on line, width=.45\linewidth}
\begin{columns}[T]
\column{.45\linewidth}
\begin{warnbox}
\small\centering Doppeltes \mintinline{c}{&&} und \mintinline{c}{||}!\newline
\end{warnbox}
%
\column{.45\linewidth}
%\hspace{\fill}
\begin{warnbox}
\small\centering Ihr wollt mir nicht allein im Dunkeln begegnen.\vspace{\fill}
\end{warnbox}
\end{columns}
%
\end{frame}

% =========================================================================== %

\begin{frame}[fragile]
%
\begin{codebox}
\begin{minted}[fontsize=\scriptsize,linenos]{c}
#include <stdio.h>
#include "lawlibrary.h"

int main () {
  int Fall_ID, Tatbestand, Rechtswidrigkeit, Schuldausschlussgrund;
  
  Fall_ID = getNewCase();
  
  Tatbestand            = checkActusReus(Fall_ID);
  Rechtswidrigkeit      = checkMensRea  (Fall_ID);
  Schuldausschlussgrund = checkDefense  (Fall_ID);
  
  if (Tatbestand && Rechtswidrigkeit && ! Schuldausschlussgrund) {
    printf("Schuldig!\n");
  } else {
    printf("Freispruch!\n");
  }
}
\end{minted}
\end{codebox}
%
\end{frame}

% =========================================================================== %

\begin{frame}{Bitweise Logik}
%
\begin{itemize}
\item Zuvor: Kombination von Wahrheitwerten
\item Jede Zahl aus Bits dargestellt, die ebenfalls Wahrheitswerte darstellen
\end{itemize}
%
\begin{tcolorbox}[title=Bitweise Logikoperatoren]
\begin{center}
\begin{table}
	\newcolumntype{C}{>{\ttfamily\centering\arraybackslash} p{.07\linewidth}}
	\newcolumntype{P}{>{         \centering\arraybackslash} p{.20\linewidth}}
\begin{tabularx}
	{\linewidth}	
	{PC|PC|PC}
Verknüpfung & \normalfont Code &
Verknüpfung & \normalfont Code &
Verknüpfung & \normalfont Code \tabcrlf

AND & \&   &   XOR      & \^{}              &   Bitshift links & {<}< \\
OR  & |    &   Negation & \textasciitilde   &   Bitshift rechts & {>}>  \\
\end{tabularx}
\end{table}
\end{center}
\end{tcolorbox}
%
%
\end{frame}

% =========================================================================== %

\begin{frame}[fragile]{Bitweise Logik -- Beispiele}
%
\tcbset{on line, width=.3\linewidth}
%
\begin{center}
%
\begin{codebox}[AND]
\begin{minted}{text}
1 1 0 0   12
1 0 1 0   10
- AND -   --
1 0 0 0    8
\end{minted}
\end{codebox}
%
\hspace{1em}
%
\begin{codebox}[OR]
\begin{minted}{text}
1 1 0 0   12
1 0 1 0   10
-- OR -   --
1 1 1 0   14
\end{minted}
\end{codebox}
%
\hspace{1em}
%
\begin{codebox}[XOR]
\begin{minted}{text}
1 1 0 0   12
1 0 1 0   10
- XOR -   --
0 1 1 0    6
\end{minted}
\end{codebox}
%
\end{center}
%
%
\tcbset{on line, width=.45\linewidth}
%
\begin{center}
%
\begin{codebox}[Bitshift left by 1]
\begin{minted}{text}
1 0 1 0 1 0 1 0   170
----- << 1 ----   ---
0 1 0 1 0 1 0 0    84
\end{minted}
\end{codebox}
%
\hspace{\fill}
%
\begin{codebox}[Bitshift right by 1]
\begin{minted}{text}
0 1 0 1 0 1 0 1    85
----- >> 1 ----   ---
0 0 1 0 1 0 1 0    42
\end{minted}
\end{codebox}
%
\end{center}
%
\end{frame}

% =========================================================================== %

\begin{frame}[fragile]{Hierarchie der Operatoren}
%
\begin{tcolorbox}%[title=Details]
%
\begin{tabularx}
	{\linewidth}
	{cc|cc}
	
	Rang & Operator            &  Rang & Operator                             \tabcrlf
	1    & \texttt{++, -{}-}   &  5    & Bitweise (\texttt{\&, |}, \ldots)    \\
	2    & \texttt{*, /, \%}   &  6    & Logische (\texttt{\&\&, ||}, \ldots) \\
	3    & \texttt{+, -}       &  7    & Zuweisungen (\texttt{=, +=}, \ldots) \\
	4    & Vergleiche (\texttt{>=, <, ==}, \ldots) \\
	
\end{tabularx}
\phantom{}\newline
%
\scriptsize \url{http://en.cppreference.com/w/c/language/operator_precedence}
\end{tcolorbox}
%
\begin{hintbox}
Im Zweifel: Klammern benutzen
\end{hintbox}
%
\end{frame}

% =========================================================================== %

\begin{frame}[fragile]{Verschachtelung}
%
Zwei Gleichwertige Code-Beispiele:
%
\begin{columns}[T]
\column{.48\linewidth}
\begin{codebox}[logische Verknüpfung]
\begin{minted}[fontsize=\scriptsize,linenos]{c}
if ((a == b) && (b != c)) {
   printf("a ist b, b nicht c\n");
}
\end{minted}
\end{codebox}
%
\column{.48\linewidth}
\begin{codebox}[Verschachtelung]
\begin{minted}[fontsize=\scriptsize,linenos]{c}
if    (a == b) {
   if (b != c) {
      printf("a ist b, b nicht c\n");
   }
}
\end{minted}
\end{codebox}
\end{columns}
%
\begin{hintbox}
Einrückungen retten Leben!
\end{hintbox}
%
\end{frame}

% =========================================================================== %

\begin{frame}[fragile]{Zwei Kurzformen mit \mintinline{c}{if}}
%
\begin{columns}[T]
\column{.45\linewidth}
\begin{tcolorbox}[title=\emph{Einzeilige} \mintinline{text}{if}-Blöcke]
\begin{itemize}
\item \{Geschweifte Klammern\} optional
\item Nicht zu empfehlen\newline
	(Fehlerquelle beim Erweitern)
\end{itemize}
\begin{codebox}
\begin{minted}[fontsize=\scriptsize,linenos]{c}
if (Ausdruck)
   Anweisung_1;
else
   Anweisung_2;
\end{minted}
\end{codebox}
\end{tcolorbox}
%
\column{.55\linewidth}
\begin{tcolorbox}[title=Bedingte Zuweisung]
%
\begin{codebox}[Syntax]
\scriptsize\mintinline{c}{Bedingung ? Wert_ja : Wert_nein}
\end{codebox}
\begin{codebox}[Beispiel]
\scriptsize\mintinline{c}{max_ab = a > b  ?  a  :  b;}
\end{codebox}
Kann manche Ausdrücke stark vereinfachen, an anderen Stellen eher verkomplizieren
\end{tcolorbox}
%
\end{columns}
%
\end{frame}

% =========================================================================== %

\begin{frame}[fragile]
%
\begin{columns}[T]
\column{.48\linewidth}
\begin{Large}
{\mintinline{c}{switch}-Blocks}
\vspace{6pt}
\end{Large}
\begin{itemize}
\item Fallunterscheidung für \mintinline{c}{int}- und \mintinline{c}{char}-Ausdrücke
\item \enquote{Erstes aus einer Liste}
\item Vergleich nur mit \emph{konstanten Ausdrücken}
\item \inC{break}: Verlässt aktuelle \{Umgebung\}
\item Ohne \inC{break} wird auch nachfolgender \inC{case} ausgeführt
\item \enquote{Fallthrough}, meist nicht gewollt
\item \inC{default}: Wie \mintinline{c}{else}.\newline
	\emph{Muss} letzter Anweisungsblock sein.
\end{itemize}
%
\column{.48\linewidth}
\begin{codebox}[Syntax]
\begin{minted}[fontsize=\scriptsize,linenos]{c}
switch (Ausdruck) {
   case Wert_1:
      Anweisungen_1;
      break;

   ...

   case Wert_n:
      Anweisungen_n;
      break;
      
   default:
      Anweisungen_sonst;
}
\end{minted}
\end{codebox}
\end{columns}
%
\end{frame}

% =========================================================================== %

\begin{frame}[fragile]
%
\begin{codebox}
\begin{minted}[fontsize=\scriptsize,linenos]{c}

char c = 'x';
switch (c) {
    case 'u':
        printf("Will mir jemand was vormachen?\n");
        break;
    case 'x':
        printf("Doch ein x.\n");
        break;
    default:
        printf("Seltsam\n");
}
\end{minted}
\end{codebox}
%
\begin{hintbox}
\small 'Einfache Anführungszeichen': ASCII-Wert (Zahlenwert) des eingeschlossenen Zeichens
\end{hintbox}
%
\end{frame}
