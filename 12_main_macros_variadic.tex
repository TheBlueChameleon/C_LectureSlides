% =========================================================================== %
% Yes. This is a document.

\documentclass[
	ngerman,
	aspectratio=169,
	table
]{beamer}

% =========================================================================== %
% Theme
\usepackage{scrlfile}
	\ReplacePackage{beamerthemeSHUR}{./sty/beamerthemeSHUR}
	\ReplacePackage{beamerinnerthemefancy}{./sty/beamerinnerthemefancy}
	\ReplacePackage{beamerouterthemedecolines}{./sty/beamerouterthemedecolines}
	\ReplacePackage{beamercolorthemechameleon}{./sty/beamercolorthemechameleon}

\usetheme[
	pageofpages=/,
	bullet=circle,
	titleline=true,
	alternativetitlepage=true,
	watermark="",
	watermarkheight=0px,
	watermarkheightmult=0
	]
{SHUR}

% =================================================================================================== %
% the usual stuff

\usepackage[utf8]{inputenc}
\usepackage[T1]{fontenc}
\usepackage{babel}
\usepackage{lmodern}
\usepackage{microtype}
\usepackage{csquotes}
\usepackage{xspace}

\usepackage{tabularx}
\usepackage{booktabs}
\usepackage{multirow}

\usepackage{color, colortbl}
\usepackage{xcolor}

\usepackage{tabto}

\usepackage{minted}
	\usemintedstyle{friendly}

\usepackage{tikz}
	\usetikzlibrary{positioning}
	\usetikzlibrary{matrix}
	\usetikzlibrary{shapes.geometric}
	\usetikzlibrary{backgrounds}
	\usetikzlibrary{calc}
	\usetikzlibrary{decorations.pathreplacing}
	\tikzstyle{every picture}+=[remember picture] 
\usepackage{adjustbox}

\usepackage[most]{tcolorbox}
	\tcbsetforeverylayer
		{colback=cyan!10!white,
		 colframe=cyan!75!black,
		 arc=0pt,
		 outer arc=0pt
		}
	\newtcolorbox{codebox}[1][Code]
		{colback=black!5!white,
		 colframe=blue!40!black,
		 title=#1,
		 leftupper=6mm
		}
	\newtcolorbox{cmdbox}[1][Kommandozeilen-Befehl]
		{colback=black,
		 coltext=white,
		 fontupper=\ttfamily ,
		 colframe=blue!40!black,
		 title=#1,
		 outer arc=0pt
		}
	\newtcolorbox{warnbox}[1][Beachte]
		{colback=black!5!white,
		 colframe=red!40!black,
		 title=#1
		}
	\newtcolorbox{hintbox}[1][Tipp]
		{colback=black!5!white,
		 colframe=green!40!black,
		 title=#1
		}%
	\newenvironment{itembox}[1][]
		{\begin{tcolorbox}[title=#1]\begin{itemize}}%
		{\end{itemize}\end{tcolorbox}}
	\newtcolorbox{doublebox}[1][.3]
		{righthand width=#1\linewidth,
		 sidebyside,
		 sidebyside gap=6mm,
		 sidebyside align=center,
		 lower separated=false}

%==============================================================================%
% GLOBAL MACROS

\newcommand*{\zB}{z.\,B. }
\newcommand*{\ua}{u.\,a. }
\newcommand*{\ie}{d.\,h. }			%% dh is already a defined macro. Couldn't find out what it does.
\newcommand*{\idR}{i.\,d.\,R. }

\newcommand*{\tabcrlf}{\\ \midrule}			% actually still allows for optional argument

\newcommand*{\inC}[1]{\mintinline{c}{#1}}

\newcommand*{\thus}{\ensuremath{\rightarrow}\xspace}
\newcommand*{\Thus}{\ensuremath{\Rightarrow}\xspace}

% =========================================================================== %

\author{Stefan Hartinger}
\title{Programmieren in C und C++}
\subtitle{Kursteil 12: Makros und variadische Funktionen}
\institute{Universität Regensburg, Fakultät Physik}
\date{Blockkurs Herbst 2021}

% =========================================================================== %

\begin{document}
% =========================================================================== %

\begin{frame}[t,plain]
\titlepage
\end{frame}

% =========================================================================== %

\begin{frame}{Script}
%
\begin{itemize}
\item Kapitel 14
	\begin{itemize}
	\item 14.1. Konstante Ausdrücke und vordefinierte Symbole
	\item 14.2. Bedingte Kompilierung
	\item 14.3. Parametrisierte Macros
	\item 14.4. Stringify-Operator \texttt{\#}
	\item 14.5. Concatenation-Operator \texttt{\#\#}
	\end{itemize}
\end{itemize}
%
\end{frame}

% =========================================================================== %

\begin{frame}[fragile]{\texttt{\#define} und der Präprozessor}
%
\begin{columns}[T]
\column{.43\linewidth}
\begin{itemize}
\item Präprozessor: Ändert Code, bevor Compiler einsetzt
\item Beispiel: \mintinline{c}{#include} -- wird durch Dateiinhalt ersetzt
\item \mintinline{c}{#define} -- Symbole anlegen (Makros)
	\begin{itemize}
	\item Konvention: ALL CAPS
	\end{itemize}
\item Vordefinierte Symbole
	\begin{itemize}
	\item \texttt{\_\_DATE\_\_}, \texttt{\_\_TIME\_\_}
	\item \texttt{\_\_LINE\_\_}, \texttt{\_\_FILE\_\_}
	\item \texttt{\_\_WIN32}, \texttt{WIN32}, \texttt{\_\_WIN32\_\_}
	\item \texttt{\_\_UNIX}, \texttt{\_\_APPLE}, ...
	\end{itemize}
\item \mintinline{c}{#if}-Konstrukt und \mintinline{c}{defined} 
\end{itemize}
%
\column{.5\linewidth}
\begin{codebox}
\begin{minted}[fontsize=\scriptsize,linenos]{c}
#include <stdio.h>

#define M_PI 3.14159265358979323846

#if defined(__WIN32)
    /* Windows-Specific Code */
#else
    /* Unix-Specific Code */
#endif

int main (void) {
   printf("pi = %lf\n", M_PI);
}
\end{minted}
\end{codebox}
\end{columns}
%
\end{frame}

% =========================================================================== %

\begin{frame}[fragile]{}
%
%
\begin{minipage}{.495\linewidth}
\begin{warnbox}[Kein Semikolon, leftupper=6mm, equal height group=grDangers1]
\begin{minted}[fontsize=\scriptsize,linenos]{c}
#include <stdio.h>

#define N 10;

int main () {
  int x = N;
  
  printf("%d", x);
  printf("%d", N);
}

\end{minted}
\end{warnbox}
%
\begin{codebox}[Expansion, leftupper=6mm, equal height group=grDangers2]
\begin{minted}[fontsize=\scriptsize]{c}
int x = 10;;
...  
printf("%d", 10;);
\end{minted}
\end{codebox}
\end{minipage}
%
%
\begin{minipage}{.495\linewidth}
\begin{warnbox}[Klammern, leftupper=6mm, equal height group=grDangers1]
\begin{minted}[fontsize=\scriptsize,linenos]{c}
#include <stdio.h>

#define SIX 1 + 5
#define NINE 8 + 1

int main () {
  printf("SIX multiplied by NINE equals"
         " %d.\n", SIX * NINE);
}
\end{minted}
\end{warnbox}
%
\begin{codebox}[Expansion, leftupper=6mm, equal height group=grDangers2]
\begin{minted}[fontsize=\scriptsize]{c}
printf("SIX multiplied by NINE equals"
         " %d.\n", 1 + 5 * 8 + 1);
\end{minted}
\end{codebox}
\end{minipage}
%
\end{frame}

% =========================================================================== %

\begin{frame}[fragile]
%
\begin{hintbox}[Empfehlung: \texttt{const} globals]
Wir haben gesehen, dass \mintinline{c}{#define}-Symbole einige Gefahren bergen. Neben den oben gezeigten Fallen bieten diese auch keine \emph{Type Safety}. Zu empfehlen sind daher globale Konstanten:
\begin{center}
\mintinline{c}{const double M_PI = 3.14159265358979323846;}
\end{center}

Der Hauptzweck von \mintinline{c}{#define}-Symbolen sind Konstruktionen mit \mintinline{c}{#if}.
\end{hintbox}
%
\end{frame}

% =========================================================================== %

\begin{frame}[fragile]
%
\begin{codebox}[Mehrzeilige Makros]
\begin{minted}[fontsize=\scriptsize,linenos]{c}
#include <stdio.h>

#define PRINT_HEADLINE \
  printf("+----------------------+\n"); \
  printf("| headline             |\n"); \
  printf("+----------------------+\n")

int main () {
  PRINT_HEADLINE;
}
\end{minted}
\end{codebox}
%
\begin{cmdbox}[Ausgabe]
\begin{minted}[fontsize=\scriptsize]{text}
+----------------------+
| headline             |
+----------------------+
\end{minted}
\end{cmdbox}
%
\end{frame}

% =========================================================================== %

\begin{frame}[fragile]
%
\adjustbox{valign=t}{\begin{minipage}{.495\linewidth}
\begin{hintbox}[Vor- und Nachteile]
Vorteile
\begin{itemize}
\item Kein Overhead für Funktionsaufruf
\item Schnellere Programme
\end{itemize}
Nachteile
\begin{itemize}
\item Größere ausführbare Dateien durch Code-Kopien
\item Unerwartete Nebenwirkungen und schwer deutbare Fehlermeldungen
\end{itemize}
\end{hintbox}
\end{minipage}}
%
%
\adjustbox{valign=t}{\begin{minipage}{.495\linewidth}
\begin{hintbox}[Vor- und Nachteile]
In Deklarations-Zeile: Funktionen als \mintinline{c}{inline} ausgezeichnen\\
Compiler setzt diese ähnlich um; Verhalten im Code aber wie bei Funktionen, ohne Nebeneffekte.
\end{hintbox}
%
\begin{codebox}[Beispiel]
\begin{minted}[fontsize=\scriptsize,linenos]{c}
#include <stdio.h>

inline void print_headline();
       void print_headline() {
  printf("+----------------------+\n");
  printf("| headline             |\n");
  printf("+----------------------+\n");
}
\end{minted}
\end{codebox}
\end{minipage}}
%
\end{frame}

% =========================================================================== %

\begin{frame}[fragile]{Parametrisierte Makros -- \enquote{Mini-Funktionen}}
%
\begin{columns}[b]
\column{.55\linewidth}
\begin{itemize}
\item Ebenfalls über \mintinline{c}{#define}
\item Ebenfalls vor Kompilation verarbeitet
\item Parameter in runden Klammern () und durch Kommata getrennt
\item Kein Leerzeichen zwischen Macro-Name und Parameterliste!
\end{itemize}
%
\begin{hintbox}
\begin{itemize}
\item Keine Semikolons (;)
\item Klammern um gesamten Ausdruck
\end{itemize}
\end{hintbox}
%
\column{.45\linewidth}
\begin{codebox}[Beispiel]
\begin{minted}[fontsize=\scriptsize,linenos]{c}
#include <stdio.h>

#define MAX(a, b) (a > b ? a : b)

int main (void) {
   int x = 3, y = 7;
   
   printf(
      "max(%d, %d) = %d\n", 
      x, y, 
      MAX(x, y)
   );
}
\end{minted}
\end{codebox}
\end{columns}
%
\end{frame}

% =========================================================================== %

\begin{frame}[fragile]
%
\adjustbox{valign=t}{\begin{minipage}{.495\linewidth}
\begin{warnbox}[Nebeneffekte, leftupper=6mm]
\begin{minted}[fontsize=\scriptsize,linenos]{c}
#include <stdio.h>

#define MAX(a, b) (a > b  ?  a : b)

int main () {
  int i = 8, j = 7;
  printf("MAX: %2d\n" , MAX(i++, j++));
  printf(" i : %2d\n", i);
  printf(" j : %2d\n", j);
}
\end{minted}
\end{warnbox}
\end{minipage}}
%
%
\adjustbox{valign=t}{\begin{minipage}{.495\linewidth}
\begin{codebox}[Expansion Zeile 7, leftupper=6mm]
\begin{minted}[fontsize=\scriptsize]{c}
  printf("MAX: %2d\n" , 
    i++ > j++ ?  i++ : j++)
  );
\end{minted}
\end{codebox}

\begin{cmdbox}[Ausgabe, leftupper=6mm]
\begin{minted}[fontsize=\scriptsize]{text}
MAX: 9
 i : 10
 j : 8
\end{minted}
\end{cmdbox}
\end{minipage}}
%
%
\begin{hintbox}
Inline-Funktionen bevorzugen!
\end{hintbox}
%
\end{frame}

% =========================================================================== %

\begin{frame}[fragile]{\emph{Stringify}-Operator}
%
\begin{itemize}
\item Oft zu Debug-Zwecken: Verwandle Code-Teil in String
\item Stringify-Makro-Operator \texttt{\#}
\end{itemize}
%
\begin{codebox}[Beispiel]
\begin{minted}[fontsize=\scriptsize, linenos]{c}
#define DebugInt(x) printf("The value of %s is %d\n", #x, x)

int myInt = 5;

DebugInt(myInt);
DebugInt(myInt + 7);
\end{minted}
\end{codebox}
%
\begin{codebox}[Ausgabe]
\begin{minted}[fontsize=\scriptsize, linenos]{text}
The value of myInt is 7
The value of myInt + 7 is 12
\end{minted}
\end{codebox}
%
\end{frame}

% =========================================================================== %

\begin{frame}[fragile]{\emph{Concatenation}-Operator}
%
\begin{itemize}
\item Verkettune zwei Code-Zeichenketten zu einem Ausdruck
\item Concatenation-Operator \texttt{\#\#}
\end{itemize}
%
\tcbset{width=.495\linewidth, height=4.8cm, on line}
%
\begin{codebox}[Beispiel: Ohne Concatenation]
\begin{minted}[fontsize=\scriptsize, linenos]{c}
typedef struct {
  *name; 
  (*function)(void);
} command;

command menuItems[] = {
  {"quit", proc_quit},
  {"help", proc_help}
};
\end{minted}
\end{codebox}
%
\begin{codebox}[Beispiel: Mit Concatenation]
\begin{minted}[fontsize=\scriptsize, linenos]{c}
typedef struct {
  *name; 
  (*function)(void);
} command;

#define COMMAND(X) {#X, proc_ ## X}

command menuItems[] = {
  COMMAND(quit),
  COMMAND(help)
};
\end{minted}
\end{codebox}
%
\end{frame}

% =========================================================================== %

\begin{frame}{Script}
%
\begin{itemize}
\item Kapitel 15
	\begin{itemize}
	\item 15.5. Variadische Funktionen
	\end{itemize}
\end{itemize}
%
\end{frame}% =========================================================================== %

\begin{frame}[fragile]{Variadische Funktionen -- Variable Zahl von Argumenten}
%
\begin{columns}[T]
\column{.45\linewidth}
\begin{itemize}
\item Vgl. \texttt{printf}: Beliebige Zahl von Parametern
\item Umsetzung über \emph{variadische Funktionen}
\item Prototyp/Funktions-Signatur enthält \texttt{...} als letztes Argument
\item Intern: Übergabe eines \mintinline{c}{void *} für Daten und eines \mintinline{c}{int *} für Datenmenge
\end{itemize}
%
\column{.55\linewidth}
\begin{codebox}[Syntax: Funktions-Signatur]
\begin{minted}[fontsize=\scriptsize]{c}
return_type funcName(feste_Parameter, ...);
\end{minted}
%
\end{codebox}
\begin{itemize}
\item Funktion muss aus den festen Parametern (mindestens einer) die anderen Parameter rekonstruieren
\item Handling über Funktionen und Makros aus \texttt{stdarg.h}
\end{itemize}
\end{columns}
%
\end{frame}

% =========================================================================== %

\begin{frame}{Variadische Argumente auslesen}
%
\begin{itemize}
\item \texttt{va\_list} -- Datentyp, enthält beide Pointer
\item \texttt{va\_start(list, count)} -- Makro. Speichert Pointer von Typ \texttt{va\_list} in Variable \texttt{list}. Parameter \texttt{count} ist Zahl der Werte in der Liste.
\item \texttt{va\_arg(list, type)}  -- Makro, liest den nächsten Wert aus Liste \texttt{list} von Typ \texttt{va\_list}; Wert selbst ist vom Typ \texttt{type}.
\item \texttt{va\_end(list)} -- Makro, gibt Speicher für Liste \texttt{list} wieder frei.
\end{itemize}
%
\end{frame}

% =========================================================================== %

\begin{frame}[fragile]
%
\begin{minipage}{.49\linewidth}
\begin{codebox}[Beispiel: Standardabweichung ...]
\begin{minted}[fontsize=\scriptsize, linenos]{c}
#include <stdio.h>
#include <stdarg.h>
#include <math.h>
 
double stddev(int count, ...) {
   double sum = 0, sum_sq = 0, num;
   
   va_list args;   
   va_start(args, count);
   
   for (int i = 0; i < count; ++i) {
      num = va_arg(args, double);
      sum    += num;
      sum_sq += num * num;
   }
   va_end(args);
   
   return sqrt(sum_sq / count - 
      (sum / count) * (sum / count));
}
\end{minted}
\end{codebox}
\end{minipage}
%
\begin{minipage}{.49\linewidth}
\begin{codebox}[... Fortsetzung (von \href{https://en.cppreference.com/w/c/variadic/va_arg}{cppreference})]
\begin{minted}[fontsize=\scriptsize, linenos, firstnumber=last]{c}
int main(void) 
{
   printf(
      "Std. deviation of list: %f\n", 
      stddev(
         4, 
         25.0, 27.3, 26.9, 25.7
      )
   );
}
\end{minted}
\end{codebox}
%
\vspace{-4pt}
\begin{hintbox}[Definition: Standardabweichung]
\footnotesize Statistisches Maß für Streuung einer Messreihe $(x_i)_{i=1,\ldots,n}$ um Mittelwert 
$\overline{x}$.
\[
\sigma = \sqrt{
	\frac
		{\sum^{n}_{i=1}(x_i - \overline{x})^2}
		{n}
}
\]
\end{hintbox}
\end{minipage}
%
\end{frame}

% =========================================================================== %

\end{document}