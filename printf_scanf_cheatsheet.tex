\documentclass[
	ngerman,
	fontsize=10pt,
	parskip=half,
	titlepage=false,
	DIV=12
]{scrartcl}


\usepackage[utf8]{inputenc}
\usepackage{babel}
\usepackage[T1]	{fontenc}
\usepackage{lmodern}
\usepackage{microtype}
\usepackage{color}
\usepackage{csquotes}

\usepackage[
	a4paper,
	landscape,
	margin=2cm
]{geometry}

\usepackage{tabularx}
\usepackage{booktabs}
\usepackage{multirow}
\usepackage{makecell}

\newcommand*{\tabcrlf}{\\ \hline}			% actually still allows for optional argument
\newcommand*{\tabsec}{\\ \cline{2-5}}
\newcommand*{\tabSec}{\\ \cline{2-3}}
\newcommand*{\SLASH}{\char`\\}

\begin{document}

\part*{Formattierte Textausgabe}
\newcolumntype{T}{>{\centering\ttfamily\arraybackslash}m{.10 \linewidth}}
\newcolumntype{S}{>{\centering\ttfamily\arraybackslash}m{.10 \linewidth}}
\newcolumntype{E}{>{\centering\ttfamily\arraybackslash}m{.15 \linewidth}}
\newcolumntype{F}{>{\centering\ttfamily\arraybackslash}m{.20 \linewidth}}
\newcolumntype{G}{>{\centering         \arraybackslash}m{.35 \linewidth}}
\newcolumntype{H}{>{\centering\ttfamily\arraybackslash}m{.35 \linewidth}}
\newcolumntype{C}{>{\centering         \arraybackslash}m{.365\linewidth}}

\begin{tabularx}
	{\linewidth}
	{|T|S|EF|C|}
	\toprule[1.5pt]
	
\normalfont \bfseries Datentyp &
	\normalfont \bfseries format string &
	\multicolumn{2}{G|}{\normalfont \bfseries Beispiele} &
	\normalfont \bfseries Anmerkung
	\tabcrlf
	
\multirow{7}{*}{ \makecell{
	\texttt{float} \\
	oder \\
	\texttt{double}
} } & 
	\%f {\normalfont ,} \%lf & 
	0.700000 &
	1234567890123456768.000000 &
	Dezimalpunkt \tabsec
	
	& \%e & 
	7.0000000e-1 &
	1.234568e+18 &
	Exponentialschreibweise mit \enquote{e} \tabsec

	& \%E & 
	7.0000000E-1 &
	1.234568E+18 &
	Exponentialschreibweise mit „E“ \tabsec
	
	& \%g & 
	0.7 &
	1.23457e+18 &
	dezimal oder Exp. mit „e“ \tabsec
	
	& \%G & 
	0.7 &
	1.23457E+18 &
	dezimal oder Exp. mit „E“ \tabsec
	
	& \%a & 
	0x1.6666666666666p-1  &
	0x1.12210f47de981p+60 &
	hexadezimal mit \enquote{a} \tabsec
	
	& \%A & 
	0X1.6666666666666P-1  &
	0X1.12210F47DE981P+60 &
	hexadezimal mit \enquote{A} \tabcrlf

long double & 
	\%Lf  & 
	\multicolumn{2}{G|}{ \normalfont wie bei \texttt{float} und \texttt{double} } &
	Variationen (\texttt{\%Le}, ...) analog zu double  \tabcrlf

\multirow{2}{*}{char} & 
	\%c & 
	\multicolumn{2}{H|}{A} &
	als ASCII-Zeichen, Werte von 0 bis 255 \tabsec
	
	& \%hhi &
	\multicolumn{2}{H|}{65} &
	als Dezimalzahl, Werte von 0 bis 255 \tabcrlf
	
\multirow{6}{*}{\texttt{int}}
	& \%d, \%i & 
	-1 &
	2147483647 &
	dezimal \tabsec
	
	& \%u & 
	4294967295 &
	2147483647 &
	dezimal, ohne Vorzeichen \tabsec
	
	& \%o & 
	37777777777 &
	17777777777 &
	oktal, ohne Vorzeichen \tabsec

	& \%x & 
	ffffffff &
	7fffffff &
	unsigned hexadezimal \tabsec

	& \%X & 
	FFFFFFFF &
	7FFFFFFF &
	unsigned hexadezimal \tabcrlf
	
long int &
	\%li, \%lx, ... &
	-1 &
	9223372036854775807 &
	alle Formen wie bei \texttt{int} \tabcrlf
	
short int &
	\%hi & 
	-1 &
	32767 &
	Variationen analog zu int \tabcrlf

\multirow{5}{*}{\makecell{alle \\ Zahlen-Typen}} &
	\%+\textit{code} &
	-1 &
	+2147483647 &
	Ausgabe wie oben, immer mit Vorzeichen \tabsec

	& \%N.M\textit{code} &
	-0.70 &
 	1234567890123456768.00 &
	Vorgabe: Mindestens Platz für N Zeichen; davon M für Nachkommastellen\tabsec

	& \%0N.M\textit{code} &
	-00.70 &
	1234567890123456768.00 &
	Vorgabe: wie oben, aber leere Stellen werden mit 0 aufgefüllt\tabcrlf
	
\makecell{
	{\ttfamily char*}, \\
	{\ttfamily char[]}, \\
	{\ttfamily unsigned~char*}
} &
	\%s &
	\multicolumn{2}{H|}{abcdefg} &
	Als Zeichenkette; alle Zeichen von der adressierten Speicherstelle bis zum ersten \texttt{NULL}-Zeichen.	\tabcrlf
	
\textit{Datentyp} *\newline
\textsf{(Pointer)}
	& \%p 
	& 0x7fffc7a9dfb8
	& (nil) 
	& Pointer auf beliebigen Datentyp, Ausgabe als Hexadezimalzahl.
	\texttt{NULL} wird als \texttt{(nil)} ausgegeben. \\
	
	\bottomrule[1.5pt]
\end{tabularx}
Diese Format-Codes funktionieren sowohl mit \texttt{printf}, \texttt{sprintf}, \texttt{snprintf},  als auch mit \texttt{fprintf}.

%Siehe auch \url{http://en.cppreference.com/w/c/io/fprintf}

\clearpage
\part*{Formattierte Texteingabe}
\newcolumntype{C}{>{\centering         \arraybackslash}m{.75\linewidth}}
\renewcommand*{\tabsec}{\\ \cline{1-2}}
\begin{tabularx}
	{\linewidth}
	{|T|S|C|}
	\toprule[1.5pt]
	
\normalfont \bfseries Pointer für Datentyp &
	\normalfont \bfseries format string &
	\normalfont \bfseries Anmerkung
	\tabcrlf
	
float &
	\%f &
	\multirow{3}{*}{
	Alle Schreibweisen (Dezimalpunkt, exponential, hexadezimal) werdenautomatisch erkannt) 
	}\tabsec

double &
	\%lf & \tabsec
	
long double &
	\%Lf & \tabcrlf
	
\multirow{2}{*}{int} 
	& \%i & 
	automatische Erkennung der Basis \tabSec
	
	& \%d & 
	dezimal \tabcrlf

\multirow{3}{*}{unsigned int} 
	& \%u & 
	vorzeichenlos, dezimal \tabSec
	
	& \%o & 
	vorzeichenlos, oktal \tabSec
	
	& \%x & 
	vorzeichenlos, hexadezimal \tabcrlf

\multirow{2}{*}{char} 
	& \%hhi 
	& Werte zwischen 0 und 255 \tabSec
	
	& \%c
	& ein Zeichen \tabcrlf

short int
	& \%hi, \%hu, ...
	& Variationen analog zu \texttt{int} und \texttt{unsigned int} \tabcrlf

long int 
	& \%li, \%lu, ...
	& Variationen analog zu \texttt{int} und \texttt{unsigned int} \tabcrlf
	

long long int 
	& \%lli, ...
	& Variationen analog zu \texttt{int} und \texttt{unsigned int} \tabcrlf
	
\multirow{7}{*}{
	\makecell{
		{\ttfamily char*}, \\
		{\ttfamily char[]}, \\
		{\ttfamily unsigned~char*}
	}
} 
	& \%s
	& Zeichenkette beliebiger Länge. Leerzeichen und Tabulatoren werden als Text-Ende interpretiert.
	\tabSec
	
	& \%Ns
	& Zeichenkette mit maximaler Länge N. Leerzeichen und Tabulatoren werden als Text-Ende interpretiert.
	\tabSec
	
	& \%[\textit{list}]s
	& Zeichenkette beliebiger Länge. Nur Zeichen, die in \textit{list} aufgeführt sind, werden akzeptiert. Der Ausdruck \textit{list} kann entweder eine Aneinanderreihung aller erlaubten Zeichen sein, oder ein Ausdruck der Art \emph{von-bis}. Alles hinter dem ersten Zeichen, das nicht in \textit{list} genannt wurde, wird ignoriert.\newline
	Beispiel 1: 
	\texttt{\%[0-9a-fA-F]} erlaubt Eingabe von Hexadezimalen Ganzzahlen\newline
	Beispiel 2:
	\texttt{\%[ \SLASH ta-z]} erlaubt Eingabe von Kleinbuchstaben, dem Tabulator und dem Leerzeichen.
	\\

	\bottomrule[1.5pt]
\end{tabularx}
Diese Format-Codes funktionieren sowohl mit \texttt{scanf}, \texttt{sscanf}, \texttt{snscanf},  als auch mit \texttt{fscanf}.


\end{document}